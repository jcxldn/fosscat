\documentclass[../../main.tex]{subfiles}

\begin{document}

\noindent This section will describe and justify the proposed classes to be used in the solution.
These classes will be represented using UML (Unified Modelling Language) diagrams.


\subsubsection{Frontend}

\noindent TypeScript, the language being used for the frontend,
is an object oriented language. Therefore I can make use of
traditional OOP principles in my solution.

\begin{comment}
TODO: finish and add? 010524

These are:

% remove this?
\begin{enumerate}
    \item \textbf{Abstraction}\\
          The process of hiding data that is not directly relevant.
          A function that operates on a class should not have to be
          concerned with the inner workings of the class, instead
          that detail is abstracted away into public functions exposed
          by the class.
    \item \textbf{Encapsulation}\\
          A process of placing data and supplemental functions
          that perform operations on the data in the same object.
          Complexities of the object's implementation such as private
          variables are abstracted away and the object only exposes public
          functions to get, set, or modify the data.

    \item \textbf{Inheritance}
\end{enumerate}
\end{comment}


% \raggedright to fix large spacing between words. (https://tex.stackexchange.com/a/676495)
\noindent \begin{longtable}{ | >{\raggedright}p{0.6\textwidth} | >{\raggedright\arraybackslash}p{0.4\textwidth} | }
    \hline
    \textbf{Class UML} & \textbf{Description}                                            \\
    \hline
    % Main class UML
    \begin{center}
        \begin{tikzpicture}
            \umlclass{Main}
            {
                \underline{Public} \\

                constructor(access\_token, refresh\_token) \\\\

                static async login(): Promise<Main> \\\\

                async getUserById(id): Promise<User> \\
                async getCheckoutById(id): Promise<Checkout> \\
                async getEntityById(id): Promise<Entity> \\
                async getFileById(id): Promise<File> \\
                async getItemById(id): Promise<Item> \\\\

                async getAllUsers(): Promise<User[]> \\
                async getAllCheckouts(): Promise<Checkout[]> \\
                async getAllEntities(): Promise<Entity[]> \\
                async getAllFiles(): Promise<File[]> \\
                async getAllItems(): Promise<Item[]> \\\\

                async modifyCheckout(checkout): Promise<Checkout> \\
                async modifyEntity(entity): Promise<Entity> \\
                async modifyItem(item): Promise<Item> \\\\

                async createCheckout(newCheckout): Promise<void> \\
                async createEntity(newEntity): Promise<void> \\
                async createItem(newItem): Promise<void> \\\\

                static async createUser(newUser): Promise<void>
            }{}
        \end{tikzpicture}
    \end{center}
    % Main class description
                       & The "Main" class is the starting point for interacting with the
    backend service. It should manage authentication, as well as allow for calls to the
    backend service to fetch, modify or remove data. Authentication-wise,
    the class should ensure that upon any invocation of an asynchronous
    function that a valid access token is available, requesting a new token
    if one is not available or has expired.
    The static login function should handle authenticating the user and should
    return an instance of Main when the user successfully logs in, allowing the user
    to make authenticated requests using the instance defined instance functions.        \\
    \hline
    % User class UML
    \begin{center}
        \begin{tikzpicture}
            \umlclass{User}
            {
                \underline{Private}\\
                id: UUID \\
                firstName: string \\
                lastName: string \\
                email: string \\
            }{
                \underline{Public}
                async getId(): Promise<UUID> \\
                async getFirstName(): Promise<string> \\
                async getLastName(): Promise<string> \\
                async getEmail(): Promise<string> \\
                async getAllCheckouts(): Promise<Checkout[]> \\
                async getCheckoutById(id): Promise<Checkout>
            }{}
        \end{tikzpicture}
    \end{center}
    % User class description
                       &                                                                 \\
    \hline
    % Checkout class UML
    \begin{center}
        \begin{tikzpicture}
            \umlclass{Checkout}
            {
                \underline{Private}\\
                id: UUID \\
                user: User\\
                takeDate: Date\\
                returnDate: Date
            }{
                \underline{Public}
                async getId(): Promise<UUID> \\
                async getUser(): Promise<User> \\
                async getTakeDate(): Promise<Date> \\
                async getReturnDate(): Promise<Date>
            }{}
        \end{tikzpicture}
    \end{center}
    % Checkout class description
                       &                                                                 \\
    \hline
    % Entity class UML
    \begin{center}
        \begin{tikzpicture}
            \umlclass{Entity}
            {
                \underline{Private}\\
                id: UUID \\
                checkouts: Checkout[]
            }{
                \underline{Public}
                async getId(): Promise<UUID> \\
                async getCheckouts(): Promise<Checkout[]> \\
                async getCheckoutById(id): Promise<Checkout>
            }{}
        \end{tikzpicture}
    \end{center}
    % Entity class description
                       &                                                                 \\
    \hline
    % Item class UML
    \begin{center}
        \begin{tikzpicture}
            \umlclass{Item}
            {
                \underline{Private}\\
                id: UUID\\
                title: string\\
                entities: Entity[]
            }{
                \underline{Public}
                async getId(): Promise<UUID> \\
                async getTitle(): Promise<string> \\
                async getEntities(): Promise<Entity[]> \\
                async getEntityById(id): Promise<Entity>
            }{}
        \end{tikzpicture}
    \end{center}
    % Item class description
                       &                                                                 \\
    \hline
    % File class UML
    \begin{center}
        \begin{tikzpicture}
            \umlclass{File}
            {
                \underline{Private}\\
                id: UUID \\
                uploader: User\\
                name: string\\
                data: ArrayBuffer
            }{
                \underline{Public}
                async getId(): Promise<UUID> \\
                async getUploader(): Promise<User> \\
                async getName(): Promise<string> \\
                async getData(): Promise<ArrayBuffer>
            }{}
        \end{tikzpicture}
    \end{center}
    % File class description
                       &                                                                 \\
    \hline
\end{longtable}


% \paragraph{Backend (Server Application)}

\pagebreak

\subsubsection{Backend}

\noindent Go, the language I plan to use for the backend, does not have classes.
Instead, go uses structs and types, which can have methods attached to them, in a similar way
to classes. I will be using these Go structs with GORM, the Go database library that
I will be using. For convenience I have denoted which struct fields should be specified as
GORM primary or foreign keys. \\

% \noindent I will also not be using any methods on these structs, with the \textit{exception of the constructor}. \\

% \raggedright to fix large spacing between words. (https://tex.stackexchange.com/a/676495)
\noindent \begin{tabular}{ | >{\raggedright}p{0.3\textwidth} | >{\raggedright\arraybackslash}p{0.7\textwidth} | }
    \hline
    \textbf{Class UML} & \textbf{Description} \\
    \hline
    % User class UML
    \begin{center}
        \begin{tikzpicture}
            \umlclass{User}{
                ID: uuid.UUID \textbf{[PK]} \\
                FirstName: string \\
                LastName: string \\
                Email: string \\
                Hash: string
            }{}
        \end{tikzpicture}
    \end{center}
    % User class description
                       &                      \\
    \hline
    % Checkout class UML
    \begin{center}
        \begin{tikzpicture}
            \umlclass{Checkout}{
                ID: uuid.UUID \textbf{[PK]} \\
                EntityID : uuid.UUID \textbf{[FK]} \\
                User: User \\
                UserID: uuid.UUID \textbf{[FK]} \\
                TakeDate time.Time \\
                ReturnDate time.time
            }{}
        \end{tikzpicture}
    \end{center}
    % Checkout class description
                       &                      \\
    \hline
    % Entity class UML
    \begin{center}
        \begin{tikzpicture}
            \umlclass{Entity}{
            ID: uuid.UUID \textbf{[PK]} \\
            ItemID: uuid.UUID \textbf{[FK]} \\
            Checkouts: []Checkout
            }{}
        \end{tikzpicture}
    \end{center}
    % Entity class description
                       &                      \\
    \hline
    % Item class UML
    \begin{center}
        \begin{tikzpicture}
            \umlclass{Item}{
            ID: uuid.UUID \textbf{[PK]} \\
            Title: string \\
            Entities: []Entity
            }{}
        \end{tikzpicture}
    \end{center}
    % Item class description
                       &                      \\
    \hline
\end{tabular}

\end{document}