\documentclass[../../main.tex]{subfiles}

\begin{document}

\noindent This section will describe and justify the proposed classes to be used in the solution.
These classes will be represented using UML (Unified Modelling Language) diagrams.

% \paragraph{Backend (Server Application)}

\noindent Go, the language I plan to use for the backend, does not have classes.
Instead, go uses structs and types, which can have methods attached to them, in a similar way
to classes. I will be using these Go structs with GORM, the Go database library that
I will be using. For convenience I have denoted which struct fields should be specified as
GORM primary or foreign keys. \\

% \noindent I will also not be using any methods on these structs, with the \textit{exception of the constructor}. \\

% \raggedright to fix large spacing between words. (https://tex.stackexchange.com/a/676495)
\noindent \begin{tabular}{ | >{\raggedright}p{0.5\textwidth} | >{\raggedright\arraybackslash}p{0.5\textwidth} | }
    \hline
    \textbf{Class UML} & \textbf{Description} \\
    \hline
    % User class UML
    \begin{center}
        \begin{tikzpicture}
            \umlclass{User}{
                ID: uuid.UUID \textbf{[PK]} \\
                FirstName: string \\
                LastName: string \\
                Email: string \\
                Hash: string
            }{}
        \end{tikzpicture}
    \end{center}
    % User class description
                       & TBD                  \\
    \hline
    % Checkout class UML
    \begin{center}
        \begin{tikzpicture}
            \umlclass{Checkout}{
                ID: uuid.UUID \textbf{[PK]} \\
                EntityID : uuid.UUID \textbf{[FK]} \\
                User: User \\
                UserID: uuid.UUID \textbf{[FK]} \\
                TakeDate time.Time \\
                ReturnDate time.time
            }{}
        \end{tikzpicture}
    \end{center}
    % Checkout class description
                       & TBD                  \\
    \hline
    % Entity class UML
    \begin{center}
        \begin{tikzpicture}
            \umlclass{Entity}{
            ID: uuid.UUID \textbf{[PK]} \\
            ItemID: uuid.UUID \textbf{[FK]} \\
            Checkouts: []Checkout
            }{}
        \end{tikzpicture}
    \end{center}
    % Entity class description
                       & TBD                  \\
    \hline
    % Item class UML
    \begin{center}
        \begin{tikzpicture}
            \umlclass{Item}{
            ID: uuid.UUID \textbf{[PK]} \\
            Title: string \\
            Entities: []Entity
            }{}
        \end{tikzpicture}
    \end{center}
    % Item class description
                       & TBD                  \\
    \hline
\end{tabular}

\end{document}