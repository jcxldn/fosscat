\documentclass[../../main.tex]{subfiles}

\begin{document}

\begin{comment}
% home child
\node (home_details) [mb_node, below of=home, xshift=2.5cm] {System information};
\node (home_stats) [mb_node, below of=home_details, xshift=-0.4cm] {Usage statistics};

% item child
\node (item_photo) [mb_node, below of=item, xshift=2cm] {Photo(s) card};
\node (item_check) [mb_node, below of=item_photo] {Check in/out};
\node (item_entity) [mb_node, below of=item_check] {Entity list};
\node (item_history) [mb_node, below of=item_entity] {Entity history};

% entity child
\node (entity_check) [mb_node, below of=entity, xshift=2cm] {Check in/out};
\node (entity_history) [mb_node, below of=entity_check] {History};

\end{comment}

\subsection{Decomposing the problem through the use of top-down design}

\noindent \\ This section will provide an overview of the proposed solution,
making use of a "top-down" approach. This is the process of breaking down the proposed
solution's various modules into smaller, more manageable chunks.
The decomposed overview should make it clear what aspects of the system exist and what
functions they perform.

\noindent \\ I have tried to present the proposed system's design in a clear manner,
outlining what modules of the system exist, their functions, and how they interact
with other modules of the system. For the frontend module, I have split each "screen" or page into it's
own module.

\subsection{Frontend screens (pages)}

\begin{tikzpicture}[node distance=2cm]
    % objects
    \node (root) [mb_node, align=center] {
        Frontend\\
        Application\\
        Screens
    };
    \node (entity) [mb_node, right of=root, xshift=2cm] {Entity};
    \node (scan) [mb_node, above of=entity] {Scan};
    \node (home) [mb_node, above of=scan] {Home Screen};
    \node (user) [mb_node, below of=entity] {User};
    \node (item) [mb_node, below of=user] {Item};
    % TODO: add about?

    % lines between objects
    \draw [line] (root) -- (scan.west);
    \draw [line] (root) -- (home.west);
    \draw [line] (root) -- (entity.west);
    \draw [line] (root) -- (user.west);
    \draw [line] (root) -- (item.west);
\end{tikzpicture}

\pagebreak

\subsection{Backend Modules}

\begin{tikzpicture}[node distance=2cm]
    % objects
    \node (root) [mb_node] {Main};

    \node (graphql) [mb_node, right of=root, xshift=2cm] {2. GraphQL API};
    \node (database) [mb_node, above of=graphql, yshift=2cm] {1. Database};
    \node (actions) [mb_node, below of=graphql, yshift=-2cm] {3. Handlers};

    % align for \\ 
    % https://tex.stackexchange.com/questions/31096/how-can-i-use-linebreak-inside-a-node-in-tikz
    \node (database_explain) [
        mb_node,
        right of=database,
        align=left,
        xshift=4.5cm,
    ] {
        The database module will be responsible\\
        for interacting with and managing the\\
        database. It's functions will include\\
        creating tables, adding, removing and\\
        updating data, as well as searching\\
        through the database.
    };

    \node (graphql_explain) [
        mb_node,
        right of=graphql,
        align=left,
        xshift=4.5cm
    ] {
        The api module will receive requests\\
        from the client and will be responsible\\
        for serving the request and calling the\\
        correct action function. The module will\\
        also check that requests to authorized\\
        routes have valid credentials before \\
        fulfilling the request.
    };

    \node (actions_explain) [
        mb_node,
        right of=actions,
        align=left,
        xshift=4.3cm,
    ] {
        The handler module will consist of the\\
        functions that fulfil each request.
    };

    % lines between objects
    \draw [line] (root) -- (database);
    \draw [line] (root) -- (graphql);
    \draw [line] (root) -- (actions);

    \draw [line] (database) -- (database_explain);
    \draw [line] (graphql) -- (graphql_explain);
    \draw [line] (actions) -- (actions_explain);


\end{tikzpicture}

\subsubsection{Database Module}

\begin{tikzpicture}[node distance=2cm]
    \node (root) [mb_node] {1. Database};

    \node (connection) [mb_node, below of=root, xshift=-6cm] {Connection};
    \node (migration) [mb_node, below of=root, xshift=-2cm] {Migration};
    \node (searching) [mb_node, below of=root, xshift=2cm] {Searching};
    \node (updating) [mb_node, below of=root, xshift=6cm] {Updating};

    % lines between objects
    \draw [line] (root) -- (connection);
    \draw [line] (root) -- (migration);
    \draw [line] (root) -- (searching);
    \draw [line] (root) -- (updating);

\end{tikzpicture}

\noindent \\ The first sub-module is the \textbf{database} module. This module
is responsible for interacting with and managing the database, in this
case an instance of PostgreSQL running on the same server as the backend.
I am making use of \underline{abstraction} in this module to
abstract away complex database calls into easier to understand
functions. These functions fall into four categories:

\begin{enumerate}
    \item \textbf{Connection} will control the connection from the
          backend service to the PostgreSQL database instance. It will be in charge
          of reading \underline{environment variables} to determine
          the address, port and authentication settings of the PostgreSQL
          instance and will attempt to make a connection. Once connected,
          it should maintain the connection until the program is terminated,
          upon which it should finish any pending actions for the database
          and \underline{gracefully} close the connection.

    \item \textbf{Migration} will control the tables and schema of the
          database, as well as setting any constraints such as
          primary/foreign keys, nullable fields and the field type.
          It should handle creating any missing tables in order to conform
          with section 2.6.1 as well as creating fields based on the
          classes/structs defined in section 2.6.2.

    \item \textbf{Searching} will be an abstracted function that should
          provide an easy way to search through the database, using table fields
          such as ID and eg. FirstName/LastName. It should handle constructing
          and executing a relevant SQL statement to find the requested object.

    \item \textbf{Updating} will enable the program to update an existing
          object in the database, overwriting the old object in-place without creating
          an additional copy. It should be an abstracted function that takes the updated
          struct (class) as a parameter and updates the corresponding entry in the database.

\end{enumerate}

\subsubsection{GraphQL API Module}

\begin{tikzpicture}[node distance=2cm]
    \node (root) [mb_node] {2. GraphQL API};

    \node (operation) [mb_node, right of=root, xshift=2cm] {Operation};
    \node (auth) [mb_node, above of=operation, yshift=2cm] {Authentication};
    \node (validation) [mb_node, below of=operation, yshift=-1cm] {Validation};

    % Authentication children
    \node (auth_middleware) [mb_node, right of=auth, xshift=2cm, yshift=2cm] {Auth. middleware};
    \node (auth_routes) [mb_node, right of=auth, xshift=2cm, align=center] {
        Auth. routes\\
        (mutations)
    };
    \node (logout) [mb_node, right of=auth_routes, xshift=2cm] {Logout};
    \node (login) [mb_node, above of=logout] {Login};
    \node (signup) [mb_node, below of=logout] {Sign up};


    % Operation children
    \node (query) [mb_node, right of=operation, yshift=1cm, xshift=2cm, align=center] {
        Query\\
        (read operation)
    };

    \node (mutation) [mb_node, below of=query, align=center] {
        Mutation\\
        (write operation,\\
        function)
    };
    % \node (handler) [mb_node, right of=query, xshift=2cm, yshift=-1cm] {Request Handler};

    % Validation children
    \node (error) [mb_node, right of=validation, xshift=2cm] {GraphQL error handling};

    % lines
    \draw [line] (root) -- (auth);
    \draw [line] (root) -- (operation);
    \draw [line] (root) -- (validation);

    % auth lines
    \draw [line] (auth) -- (auth_middleware);
    \draw [line] (auth) -- (auth_routes);
    \draw [line] (auth_routes) -- (login);
    \draw [line] (auth_routes) -- (logout);
    \draw [line] (auth_routes) -- (signup);

    % operation lines
    \draw [line] (operation) -- (query);
    \draw [line] (operation) -- (mutation);

    %\draw [line] (query) -- (handler);
    %\draw [line] (mutation) -- (handler);

    % validation lines
    \draw [line] (validation) -- (error);

\end{tikzpicture}

\noindent \\ The second sub-module is the \textbf{GraphQL API} module.
This module will hold all functions related to the operation of the
graphQL API, with the exception of the request handlers themselves.
The request handlers have been \underline{split} into a separate
sub-module so that they can be used without having to invoke them
via GraphQL.

\noindent \\ The functions in this module fall into three categories:

\begin{enumerate}
    \item \textbf{Authentication} will handle calls to the three
          authentication routes, as well as registering a middleware function.
          \begin{enumerate}
              \item Authentication middleware\\
                    This middleware will ensure that requests to "authenticated"
                    routes (routes requiring authorization) are accompanied by a
                    valid authentication token. This algorithm is thoroughly
                    documented in \underline{Section 2.5.1}.

              \item Authentication routes\\
                    These routes will serve as a method for a client
                    to "log-in" to the system. To log-in, client will send
                    credentials which will be validated on the server. If
                    the credentials are valid the client will be issued
                    two tokens, an access token and a refresh token.
                    More details about this approach using tokens is explained
                    in \underline{Section 2.1, No. 4, JSON Web Tokens}
          \end{enumerate}

    \item \textbf{Operation} will handle all other GraphQL requests,
          such as a request to fetch user information.
          There are two types of requests in GraphQL which each require
          a different type of handler, queries and mutations.
          More information about queries and mutations is available in
          \underline {Section 2.1, No. 3, GraphQL}.\\
          \begin{enumerate}
              \item Query\\
                    \resizebox{\textwidth}{!}{
                        \begin{tikzpicture}[node distance=2cm]
                            \node (query) [mb_node, align=center] {Query\\(read operation)};
                            % operation query children
                            % center 
                            \node (query_items) [mb_node_small, below of=query] {Get items};
                            % above
                            \node (query_file) [mb_node_small, left of=query_items, xshift=-0.5cm] {Get files};
                            \node (query_entity) [mb_node_small, left of=query_file, xshift=-0.5cm] {Get entities};
                            \node (query_checkout) [mb_node_small, left of=query_entity, xshift=-1cm] {Get checkouts};
                            % below
                            \node (query_users) [mb_node_small, right of=query_items, xshift=0.5cm] {Get users};
                            \node (query_me) [mb_node_small, right of=query_users, xshift=0.5cm] {Get my user};
                            \node (query_refresh) [mb_node_small, right of=query_me, xshift=1.5cm] {Get refresh token};
                            % lines
                            \draw [line] (query) -- (query_items.north);
                            \draw [line] (query) -- (query_file.north);
                            \draw [line] (query) -- (query_entity.north);
                            \draw [line] (query) -- (query_checkout.north);
                            \draw [line] (query) -- (query_users.north);
                            \draw [line] (query) -- (query_me.north);
                            \draw [line] (query) -- (query_refresh.north);
                        \end{tikzpicture}}

              \item Mutation\\
                    \resizebox{\textwidth}{!}{
                        \begin{tikzpicture}[node distance=2cm]
                            \node (mutation) [mb_node, align=center] {Mutation\\(write operation, function)};
                            % operation mutation children
                            \node (mutation_item) [mb_node, below of=mutation, xshift=2.5cm] {Create/modify item};
                            \node (mutation_user) [mb_node, right of=mutation_item, xshift=2.5cm] {Create/modify user};
                            \node (mutation_entity) [mb_node, left of=mutation_item, xshift=-2.5cm] {Create/modify entity};
                            \node (mutation_checkout) [mb_node, left of=mutation_entity, xshift=-3cm] {Create/modify checkout};
                            %lines
                            \draw [line] (mutation) -- (mutation_item.north);
                            \draw [line] (mutation) -- (mutation_user.north);
                            \draw [line] (mutation) -- (mutation_entity.north);
                            \draw [line] (mutation) -- (mutation_checkout.north);
                        \end{tikzpicture}}
          \end{enumerate}

    \item \textbf{Validation} will handle error handling for GraphQL-related errors
          and performing preliminary validation on requests.\\\\
          It will be responsible for catching any errors that occur before the
          request handler is called (such as a GraphQL validation error when the
          request does not match the schema) and passing them to an appropriate
          error handler (detailed below). For queries or mutations that take arguments,
          this module should validate the inputs before passing them
          to their respective handler, returning a human-readable error
          message to to the GraphQL caller (eg. the frontend application) if needed.\\
          This sub-module will also consist of the \textbf{GraphQL Schema}, which defines
          the structure and rules of the GraphQL API.
\end{enumerate}

\subsubsection{Handler Module}

\noindent \\ The final sub-module is the Handler module. This module consists of
the functions that perform the actions required to fulfil a request.
Each function in this module should fully validate any and all inputs, as opposed
to the preliminary validation against the GraphQL schema done when calling using GraphQL.
I have designed the system in this way so that the request handlers are not reliant
on the availability of the GraphQL module. This increases the solution's resiliency
as it allows for the addition of other methods of calling handlers in the future.

\noindent \\ De-coupling the request handlers from GraphQL also allows them to be
tested \textbf{independently} of the GraphQL module. This allows for easier
testing of the solution as each module can be tested individually,  avoiding
a circular dependency which would make it more difficult to track down any problems
encountered during the development of the solution.

\subparagraph{Error Handlers\\}

\noindent \\ This module will also consist of error handlers for the request handlers.
Error handlers are middleware functions that run before and after a request handler to:

\begin{enumerate}
    \item \textbf{Ensure that all prerequisites for a request to be handled, such
              as any required route authorization, are present and valid.}

          This ensures that the solution can handle edge case scenarios such as if:

          \begin{enumerate}
              \item The authentication middleware was to malfunction.

              \item The solution cannot not access data necessary to fulfil the desired request, such as if the database was not available.
          \end{enumerate}

          \noindent In both of these cases, instead of trying to fulfil the request
          which could cause undefined
          behaviour (such as a crash), the solution will check
          to make sure that the prerequisites are available before attempting
          to fulfil the request, preventing such errors from occurring.
          This helps to improve the resiliency of the solution by making it
          more tolerant of module or dependency failures.


    \item \textbf{Ensure that eny errors encountered during the
              fulfilment of a request are\\ handled correctly.}

\end{enumerate}

\noindent The use of error handlers in the proposed solution serves to
help fulfil success criteria objective 2b. (\textit{To produce a solution
    that returns friendly, understandable errors when an error occurs})

\end{document}