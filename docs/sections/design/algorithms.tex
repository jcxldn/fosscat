\documentclass[../../main.tex]{subfiles}

\begin{document}

\noindent My solution will incorpoate a number of algorithms for the program's key tasks.
These algorithms, while some may be complex to implement due to the number of possible
output states, all are simple to represent with a flowchart or pseudocode.

\noindent \\ The algorithms will be explained along with either a flowchart or by
using pseudocode inline with the explanation. At this stage, in order to simplify
the representations of the algorithms,  I will simplify their various error states
into a single error action. These error states will be documented later on as they
are developed further.

\begin{dummyenv}
    \begin{wrapfigure}{r}[64pt]{0.5\textwidth}
        \begin{framed}{}
            \resizebox{\textwidth}{!}{\begin{tikzpicture}[node distance=2cm]
                    %\node (root) [mb_node] {Application};
                    %\node (scan) [mb_node, below of=root] {Scan};
                    %\draw [line] (root) -- (scan);

                    % Objects
                    \node (start) [startstop] {Start};
                    \node[align=center] (inp1) [io, below of=start] {Read Authorization\\ Header};
                    \node (dec1) [decision, below of=inp1, yshift=-1cm] {Header exists?};
                    %\node (out1) [io, right of=dec1, xshift=3cm] {Display error};

                    % Header exists: no
                    \node[align=center] (no-dec1) [decision, right of=dec1, xshift=3cm] {Does request \\require \\authorization?};
                    %\node (no-dec1-pro1) [process, right of=no-dec1, xshift=3cm] {Request Handler};
                    \node[inner sep=0pt] (positioning1) [right of=no-dec1, xshift=2cm] {};
                    \node[inner sep=0pt] (positioning2) [below of=positioning1, yshift=-9cm] {};

                    % Header exists: yes
                    \node (pro1) [process, below of=dec1, yshift=-1cm] {Parse token};
                    \node (dec2) [decision, below of=pro1, yshift=-1cm] {Was successful?};
                    \node (out2) [io, right of=dec2, xshift=3cm] {Display error};

                    \node[inner sep=0pt] (positioning3) [below of=out2, yshift=-5cm] {};
                    %\node (end) [startstop, xshift=1cm] {End};

                    \node (pro2) [process, below of=dec2, yshift=-1cm] {Set user context};
                    \node (pro3) [process, below of=pro2] {Request Handler};

                    \node (end) [startstop, below of=pro3] {End};

                    % Lines
                    \draw [arrow] (start) -- (inp1);
                    \draw [arrow] (inp1) -- (dec1);

                    \draw [arrow] (dec1) -- node[anchor=south] {No} (no-dec1);
                    \draw [arrow] (no-dec1) -- node[anchor=east] {Yes} (out2);
                    %\draw [arrow] (no-dec1) -- node[anchor=east] {Yes} (no-dec1-pro1);

                    \draw [arrow] (dec1) -- node[anchor=east] {Yes} (pro1);
                    \draw [arrow] (pro1) -- (dec2);

                    \draw [arrow] (dec2) -- node[anchor=south] {No} (out2);
                    \draw [arrow] (dec2) -- node[anchor=east] {Yes} (pro2);
                    \draw [arrow] (pro2) -- (pro3);
                    \draw [arrow] (pro3) -- (end);

                    % Line between no-dec1 and pro3
                    \draw [thick,-,>=stealth] (no-dec1) edge node[anchor=south] {No} (positioning1) (positioning1) edge (positioning2); % (positioning2) edge (pro3);
                    \draw [arrow] (positioning2) -- (pro3);

                    % Line between out2 and end
                    \draw [thick,-,>=stealth] (out2) -- (positioning3);
                    \draw [arrow] (positioning3) -- (end);



                    %\draw [arrow] (dec1) -- node[anchor=south] {No} (out1);
                    %\draw [arrow] (out1) -- (end);


                \end{tikzpicture}
            }

            \caption{
                \centering
                Authorization Middleware Flowchart
            }
        \end{framed}
        \label{fig:wrapfig}
    \end{wrapfigure}

    \setlength{\parindent}{0mm}
    \begin{minipage}[t]{1\linewidth}
        \paragraph{} % spacing

        \subsection{Authorization Middleware}

        \subparagraph{Background}

        \noindent \\\\ In order to fulfil success criteria objective 7b (\textit{To produce a solution that supports multiple
            distinct user accounts, providing a secure authentication system to manage access}), the server component of
        the program (the "back-end") needs to support the concept of authenticated routes, or actions.
        This means that certain actions, such as checking in an item or creating a new item, will require authorization
        before the action can be carried out. This prevents unauthorized users from modifying the entries in the system
        and brings the system closer to matching the feature-set of the competing systems researched previously. This
        also makes the system more robust, as to fulfil this objective the system will be designed in such a way
        to allow for the possibility of adding an "audit log", to keep track of what actions where performed by
        a user at a specific time. This would further increase accountability and make the system better suited to larger
        deployments. However, as noted in the research section 1.3.2, (\textit{Features to be incorporated into the solution}),
        this feature will not be implemented at this time as it was not deemed a critical feature for our stakeholders.

    \end{minipage}
\end{dummyenv}

\subparagraph{Usage}

\noindent \\\\ When the client application (the "front-end") wishes to\\call a authenticated route, or action, it
should add an\\additional header to the request before it is sent to the server. The client will add an "Authorization"
header, a reserved HTTP header used for the purpose of sending an authentication token with a request.

\noindent \\ As the server starts, it will register the implementation of this algorithm as "middleware", meaning
it will be carried out before any request is handled by the rest of the program.

\subparagraph{Algorithm}

\noindent \\\\ When the server recieves a request, before the appropriate request "handler" is called to carry
out the request, this algorithm will be called. The algorithm will check to see if the "Authorization" header
was set in the request. If it was not set, and the request being called is an authenticated route, the
algorithm will return an error to the client and the request handler will not be called. Otherwise, the algorithm
will try and parse the token specified in the header. If the token parses successfully, the token is valid and the
details of the user the token belongs to will be sent to the request handler and the latter will be executed.
If the token did not parse successfully, an error will be returned.


\subparagraph{Justification for algorithm}

\noindent \\\\ This algorithm exists to facilitate the implementation of success criteria 7b. I have chosen
to make this algorithm serve as "middleware" as it reduces code duplication. This means that instead of copying
a procedure to check and authenticate a user for \underline{every} action that requires it, this algorithm
can be defined \textit{once} in the codebase and automatically executed before every request.


\begin{dummyenv}
    \begin{wrapfigure}{r}[64pt]{0.5\textwidth}
        \begin{framed}{}
            \resizebox{\textwidth}{!}{
                \begin{tikzpicture}[node distance=2cm]

                    % Objects
                    \node (start) [startstop] {Start};
                    \node (pro1) [process, below of=start] {Execute Auth Middleware};
                    \node[align=center] (dec1) [decision, below of=pro1, yshift=-1cm] {Was\\context\\received?};

                    % > Context was not received
                    \node (out1) [io, below of=dec1, yshift=-2cm] {Display error};
                    \node (end) [startstop, below of=out1, yshift=-4cm] {End};

                    % > Context was received
                    \node (pro2) [process, right of=dec1, xshift=5cm] {Read request body};
                    \node[align=center] (dec2) [decision, below of=pro2, yshift=-2cm] {Was\\upload field\\set?};

                    \node (pro3) [process, below of=dec2, yshift=-2cm] {Process files to be uploaded};

                    % Lines
                    \draw [arrow] (start) -- (pro1);
                    \draw [arrow] (pro1) -- (dec1);

                    \draw [arrow] (dec1) -- node[anchor=east] {No} (out1);
                    \draw [arrow] (out1) -- (end);

                    \draw [arrow] (dec1) -- node[anchor=south] {Yes} (pro2);
                    \draw [arrow] (pro2) -- (dec2);

                    \draw [arrow] (dec2) -- node[anchor=south] {No} (out1);
                    \draw [arrow] (dec2) -- node[anchor=east] {Yes} (pro3);

                    \draw [arrow] (pro3) -- (end);

                \end{tikzpicture}
            }

            \caption{
                \centering
                Request Handler Flowchart
            }
        \end{framed}
        \label{fig:wrapfig}
    \end{wrapfigure}

    \setlength{\parindent}{0mm}
    \begin{minipage}[t]{1\linewidth}
        \paragraph{} % spacing

        \subsection{Uploading Files - Request Handler}

        \subparagraph{Background}

        \noindent \\\\ In order to fulfil success criteria objective 2a (\textit{To produce a solution that supports
            uploading thumbnail images for items stored in the system.}), the server component should support uploading
        files. Since the client will likely communicate with the server over HTTP, the convention for uploading a file
        over HTTP should be used. This convention involves using a HTTP \lstinline{POST} request with a HTTP form as
        a request body. To make use of this convention, the client would send such a request where a field in the form
        was set to the contents of the file.\\

        \subparagraph{Usage}

        \noindent \\\\ When the user tries to upload a image, the client program should send a POST request to the
        server. The client should set a form field to the contents of the file. This algorithm should validate the
        request by checking authorization as noted above and then call another algorithm to process the file(s) to
        be uploaded.

    \end{minipage}
\end{dummyenv}

\subparagraph{Justification for algorithm}

\noindent \\\\ This algorithm exists to facilitate the implementation of sucess criteria objective 2a.
I have chosen to split off file processing to another algorithm, detailed below.

\pagebreak


\begin{dummyenv}
    \begin{wrapfigure}{r}[64pt]{0.5\textwidth}
        \begin{framed}{}
            \begin{lstlisting}
                    function processFiles(files: byte[])
                        fileIDs = []
                        for i = 0 to len(files)
                            if isImage(file)
                                id = addToDatabase(file)
                                fileIDs.append(id)
                            else
                                print "unsupported file type"
                            endif
                        endfor
                    endfunction
            \end{lstlisting}
            \caption{
                \centering
                File Processing Pseudocode
            }
        \end{framed}
        \label{fig:wrapfig}
    \end{wrapfigure}

    \setlength{\parindent}{0mm}
    \begin{minipage}[t]{1\linewidth}
        \paragraph{} % spacing

        \subsection{Uploading Files - File Processing}

        \subparagraph{Background}

        \noindent \\\\ This algorithm support the previous algorithm in fulfilling success criteria objective 2a
        (\textit{To produce a solution that supports uploading thumbnail images for items stored in the system.}).

        \noindent \\ When files are received by the server, they needs to be processed. This processing will involve
        two actions, both of which are algorithms in their own right:\\
    \end{minipage}
\end{dummyenv}


\begin{enumerate}
    \item Checking if the file is an image
    \item Creating a "File" object and storing the image in the database
\end{enumerate}

\noindent These two steps are represented above in pseudocode.
Since neither this algorithm or the two sub-algorithms noted above have been fully implemented at this stage,
I have omitted both error handling and specific implementation details (which I do not yet know) from the
pseudocode.

\subparagraph{Justification for algorithm}

\noindent \\ This algorithm supports the above algorithm (\textit{Uploading Files - Request Handler}) in
fulfilling success criteria objective 2a. I have chosen to split this algorithm from the one above due reduce
the complexity of my algorithms and so that each algorithm can be used independently of the other in the future.
I hope to keep my implementation of this solution as modular as possible to aid future development and I believe
splitting up these algorithms is one way to accomplish this.

\end{document}