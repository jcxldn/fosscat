\documentclass[../../main.tex]{subfiles}

\begin{document}

\begin{center}
    \begin{tikzpicture}
        % Nodes
        \pic {entity={user}{User}{
                    ID: UUID, PK \\
                    FirstName: text \\
                    LastName: text \\
                    Email: text \\
                    Hash: text \\
                }};
        \pic[below=3em of user] {entity={checkout}{Checkout}{
                    ID: UUID, PK \\
                    EntityID: UUID, FK \\
                    UserID: UUID, FK \\
                    TakeDate: time \\
                    ReturnDate: time
                }};
        \pic[left=6em of checkout, text width=4.5cm] {entity={entity}{Entity}{
                    ID: UUID, PK \\
                    ItemID: UUID, FK \\
                    \mbox{Checkouts: Checkout[]} % no line break
                }};
        \pic[above=5em of entity] {entity={item}{Item}{
                    ID: UUID, PK \\
                    Title: text \\
                    Entities: Entity[]
                }};
        % Empty nodes for pathfinding
        % https://tex.stackexchange.com/questions/286791/invisible-states-in-automata-with-tikz
        \node[state, draw=none] (user1) [right=3em of user] {};
        \node[state, draw=none] (item1) [left=3em of item] {};
        % Lines
        %\draw[one to omany] (user.east) -| ($(user1.west)!0!(checkout.west)$) |- node[below]{\footnotesize is in} (checkout.east);
        \draw[one to omany] (user.east) -| ($(user1.west)!0!(checkout.west)$) |- node[below]{} (checkout.east);
        \draw[one to omany] (entity.east) -- node[below]{} (checkout.west);
        \draw[one to omany] (item.west) -| ($(item1.east)!0!(item.east)$) |- node[below]{} (entity.west);
    \end{tikzpicture}
\end{center}

\end{document}