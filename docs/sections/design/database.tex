\documentclass[../../main.tex]{subfiles}

\begin{document}

\paragraph{} % spacing

% Dummy env to escape wrapfig
\begin{dummyenv}
    % 50pt is overhang
    \begin{wrapfigure}{r}[50pt]{0.6\textwidth}
        \begin{framed}
            \begin{tikzpicture}[scale=0.8, every node/.style={scale=0.8}]
                % Nodes
                \pic {entity={user}{User}{
                            ID: UUID, PK \\
                            FirstName: text \\
                            LastName: text \\
                            Email: text \\
                            Hash: text \\
                        }};
                \pic[right=6em of user] {entity={checkout}{Checkout}{
                            ID: UUID, PK \\
                            EntityID: UUID, FK \\
                            UserID: UUID, FK \\
                            TakeDate: time \\
                            ReturnDate: time
                        }};
                \pic[below=6em of checkout, text width=4.5cm] {entity={entity}{Entity}{
                            ID: UUID, PK \\
                            ItemID: UUID, FK \\
                            \mbox{Checkouts: Checkout[]} % no line break
                        }};
                \pic[left=5em of entity] {entity={item}{Item}{
                            ID: UUID, PK \\
                            Title: text \\
                            Entities: Entity[]
                        }};

                \pic[below=6em of entity] {entity={entity_file}{File for Entity}{
                            EntityID: UUID, PK \\
                            FileID: UUID, PK
                        }};

                \pic[left=6em of entity_file] {entity={file}{File}{
                            ID: UUID, PK \\
                            UploaderID: UUID, FK (User) \\
                            Name: text \\
                            Type: enum \\
                            Data: byte
                        }};
                % Empty nodes for pathfinding
                % https://tex.stackexchange.com/questions/286791/invisible-states-in-automata-with-tikz

                % Lines
                %\draw[one to omany] (user.east) -| ($(user1.west)!0!(checkout.west)$) |- node[below]{\footnotesize is in} (checkout.east);
                \draw[one to many] (user.east) -- (checkout.west);
                \draw[one to many] (entity.north) -- (checkout.south);
                \draw[one to many] (item.east) -- (entity.west);
                \draw[one to many] (entity.south) -- (entity_file.north);
                \draw[one to many] (file.east) -- (entity_file.west);
            \end{tikzpicture}

            \caption{ERD of the proposed system}

        \end{framed}
        \label{fig:wrapfig}
    \end{wrapfigure}

    \noindent \\ For the database, I have proposed the use of
    PostgreSQL, more information about which is available in
    \textit{Section 1.4.2.4} - \\
    \textit{Server Container Software Requirements}.

    \noindent \\ PostgreSQL is an implementation of an object-relational
    database. This means that it has support for features like
    \textbf{referential integrity} as well using an object-oriented
    design.

    \subparagraph{Entity Relationship Diagram}

    \noindent \\\\ On the right-hand side is the entity relationship
    diagram (ERD) for the proposed solution. There are five entities
    in the database and the ERD shows the relationships between them.

    \subparagraph{Referential integrity}

    \noindent \\\\ Referential integrity is a process that ensures
    consistency between related tables in a relational database
    by preventing inconsistencies from appearing in said database.
    This is done by preventing invalid records from being added
    to the database, for

\end{dummyenv}

\noindent example if a foreign key field was not either
null or a valid primary key from a related table.
\noindent \\ If this condition was
to occur, an error would be raised by the database system.
This helps to maintain data integrity and prevent
orphaned records, where a record with a foreign key attribute
references an invalid table for it's primary key.

\subparagraph{Normalisation}

\noindent \\ Normalisation is the process of organising a database
to reduce data duplication, improve data accuracy (by reducing update
/deletion anomalies) and to ensure consistency. This is done by applying
a set of rules which must be fulfilled in order for a database to be
considered normalised. Normalisation also helps to make SQL queries
simpler, which makes them execute more efficiently.

\end{document}