\documentclass[../../main.tex]{subfiles}

\begin{document}

In order to fulfill \textit{SMART Objective 2} of my success criteria, my solution needs to provide a user interface
that is intuitive and easy to use. In order to accomplish this, the user interface needs
to incorporate certain usability features to enable users to use the program effectively.

\noindent \\ The first usability feature I will incorporate is that the user interface will
be high-contrast. Following the principles of \textbf{Material Design}, I will create a simple
user interface with simple but contrasting colours to make elements such as text
clearly stand out. This makes the program more accessible to those who find are colourblind
or find it difficult to distinguish elements when the background is of a similar colour.

\noindent \\ To accomodate users with light sensitivity, I will allow for the program to be
used with a "dark" theme instead of the usual "light" theme. This has the added benefit of
reducing eye strain, especially when the solution is used in areas with low light.
In order to achieve this, I will use a feature built into Material Design which allows
for elements automatically switch their colour scheme to match the desired theme. This
means that this feature can be implemented with relatively little added complexity.

\noindent \\ To ensure that those who are hearing-impaired can still use the program,
the progeam will not rely on any audio cues or sound effects.

\end{document}