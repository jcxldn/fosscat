\documentclass[../../main.tex]{subfiles}

\begin{document}

\subparagraph{Benefits and drawbacks of library usage}

\noindent \\\\ Libraries are collections of modular programs that are
designed to be inserted into another program. Libraries are beneficial
to the developer of a program as:

\begin{enumerate}[label=(\alph*)]
    \item \textbf{Libraries are usually fully tested by their developer}
          which allows the developer of a program to import the library with
          the knowledge that it will work as expected, reducing the scope
          for potential issues.

    \item \textbf{Libraries are pre-built packages that can be re-used},
          allowing the developer to leverage existing code instead of spending
          development time re-implementing something that already exists.

    \item \textbf{Libraries usually have excellent code quality}
          as they are often maintained by teams of developers with strict
          code style guidelines. This ensures a high level of \textit{code quality}
          which makes the library easier to understand and work with.
\end{enumerate}

\noindent In order to save time during the development phase in order to meet
the deadline set by the stakeholders, I will be making use of
several libraries.

\noindent \\ However, libraries are not without their drawbacks, namely:

\begin{enumerate}[label=(\alph*)]
    \item \textbf{You become reliant on the developer(s) of a library}
          to update it to add new features, fix bugs, or fix security issues.

    \item \textbf{They may be missing critical functionality for your solution.}
\end{enumerate}

\noindent \\ However, these issues can be offset by the prominence of
\underline{open-source libraries}, where anyone can modify
and re-publish the code.


\subparagraph{Libraries to be utilized in the proposed solution}

\noindent \\\\ In order to address some of the concerns highlighted above,
I will be using open-source libraries where possible.
A list of the libraries I wish to use in the proposed solution is below,
listing a brief summary for each inclusion.
Justification for each library (dependency) is available in
\underline{Section 1.4.2, Frontend Software Requirements}.

\paragraph{} % spacing 

\begin{tabular}{ |p{0.2\textwidth}|p{0.2\textwidth}|p{0.5\textwidth}| }
    \hline
    \textbf{Component} & \textbf{Library name}       & \textbf{Summary}                                                    \\
    \hline
    Backend            & gqlgen                      & Go GraphQL server library                                           \\
    \hline
    Backend            & GORM                        & Go ORM library (to interface with a database)                       \\
    \hline
    Frontend           & React,\newline React Native & UI framework for mobile                                             \\
    \hline
    Frontend           & Expo                        & Platform for creating universal mobile apps using React Native      \\
    \hline
    Frontend           & jwt-decode                  & Library to decode jwt tokens.\newline (Note: this is built into Go) \\
    \hline
    Frontent           & graphql                     & Reference graphql client implementation                             \\
    \hline
\end{tabular}

\end{document}