\documentclass[../main.tex]{subfiles}

\begin{document}

\section{Summary of requirements met}

\begin{longtable}{ | >{\raggedright}p{0.8\textwidth} | >{\raggedright\arraybackslash}p{0.2\textwidth} | }
    \hline
    \textbf{SMART Objective}                                      & \textbf{Has it been met?} \\
    \hline
    1. To produce a solution capable of persistently
    cataloguing library books and other assets for a school.
                                                                  & Partially (backend works) \\
    \hline
    2. To produce an intuitive and easy to use solution
                                                                  & No                        \\
    \hline
    2a. To produce a solution that returns friendly, understandable errors when
    an error occurs                                               & Yes                       \\
    \hline
    2b. To produce a solution that supports uploading thumbnail images for items
    stored in the system.                                         & Partially (unfinished UI) \\
    \hline
    3. To produce a solution that features a fully searchable catalogue
                                                                  & No (time constraints)     \\
    \hline
    4. To produce a solution including a server component which stores data and
    manages access to the system                                  & Yes                       \\
    \hline
    5. To produce a solution containing a user interface that can be accessed via a
    mobile device                                                 & Yes (although limited)    \\
    \hline
    6. To produce a solution that is capable of efficiently storing data about multiple
    copies of an item, such as multiple copies of the same book   & Yes                       \\
    \hline
    7. To produce a solution capable of supporting multiple users & Yes                       \\
    \hline
    7a. To produce a solution that can handle multiple users interacting with the
    system simultaneously                                         & Yes                       \\
    \hline
    7b. To produce a solution that supports multiple distinct user accounts,
    providing a secure authentication system to manage access     & Yes*                      \\
    \hline
    8. To produce a solution that has the ability to "check out" items to a specific user,
    viewable from the user interface.                             & Partially (unfinished UI) \\
    \hline
\end{longtable}

\section{Evaluation of requirements}

\subsection{SMART Objective 1}

\textit{To produce a solution capable of persistently cataloguing
    library books and other assets for a school.}

\noindent \\ \textbf{SMART Objective Met?:} Partially

\begin{comment}
1. Describe what objective was
2. Justify with tests
3. Summary if you feel it did it.
\end{comment}

\noindent \\ This objective was the core objective of the solution,
as having a working solution was a key in order to be able to meet
my stakeholder's requirements.

%\noindent \\ TODO: [testing] to back up.

\noindent \\ I believe that I greatly underestimated the amount of work required
to create a working "frontend" mobile application. While I do have
experience with Expo, React and React Native, my lack of thorough planning
for the frontend screens during the design section made me unsure which UI elements
to include in the implementation as I did not specify this in adequate detail.
I believe that this contributed to me getting stuck when implementing the frontend in
subsequent iterations.

\noindent \\ Going forward I would revisit the design section to improve Section 2.3
(User Interface Design) as this the weakest part of my plan. I should have made better use
of decomposition to split the various "screens" into smaller, more well-defined chunks.
As can be seen in \textit{Section 2.4.2} (Frontend Screens), while I did split the problem
into individual screens, as well as plotting a rough user interface mockup in
\textit{Section 2.3} (User Interface Design), I did not plan the interface, specifically
which exact UI elements should be used, in sufficient detail to implement the solution
effectively. This was compounded by the fact that React, and by extension React Native,
is very component heavy, as opposed to simpler interface tools such as Tkinter.
I believe that if I had sufficiently planned what UI elements should have been used
in the Design section, that I would not have gotten stuck pondering what elements to use
and would not have ended up with a half-finished frontend.

% This was because I only noted
%a rough user interface as opposed to planning every detail which may have made it easier
%to implement in Section 3.

% wasn't sure what elements needed


\subsection{SMART Objective 2}

\textit{To produce an intuitive and easy to use solution.}

\noindent \\ \textbf{SMART Objective Met?:} No

\noindent \\ I have decided that this SMART objective has not been met due to the lack of a
finished frontend application. While the backend has been completed to a resonable standard,
with all functionality being adequately tested, the same cannot be said for the frontend
application.

\subparagraph{Why was the solution not intuitive and easy to use?}

\noindent \\\\ While I did manage to adhere to and implement all three of my usability
features (\textit{Section 2.3.1}), I feel that the solution is not intuitive as, in it's
current (unfinished) state, it is not clear how the user should navigate through the
frontend application. When I was initially designing this solution, I pondered having
tutorial screens similar to those of big-brand apps like YouTube and Spotify, before
scrapping the idea as I was unsure about where to even start implementing such a feature.

\noindent \\ To that end, given the limited and segmented functionality of the program, (segmented as
while features such as image uploading are implemented in the client the feature is hidden
in a debug menu and is not fully implemented into the user-facing interface) I cannot
in good conscience call the current solution intuitive and easy to use.

\subsubsection{SMART Objective 2a}

\textit{To produce a solution that returns friendly, understandable errors when
    an error occurs}

\noindent \\ \textbf{SMART Objective Met?:} Yes

\noindent \\ I have chosen to mark this sub-objective as completed as the backend
component of the solution has extensive error handling, and does indeed return
user-friendly, understandable errors to any client accessing it through GraphQL.
In addition, the function handlers themselves (\textit{Section 2.4.3.3 - Handler Module})
have error handling and are responsible for generating the human-readable error message
if something unexpected occurs, meaning that even if a handler was invoked directly,
for example if the system was to be extended or modified, this sub-objective would still be fulfilled.

\subsubsection{SMART Objective 2b}

\textit{To produce a solution that supports uploading thumbnail images for items
    stored in the system.}

\noindent \\ \textbf{SMART Objective Met?:} Partially

\noindent \\ As was touched on the evaluation of SMART Objective 1, while image uploading
is implemented in both the backend and frontend, the only way to access it from the frontend
is via a debug menu accessed from the About screen. Due to this, I must note that this SMART
objective was only partially met.

\noindent \\ In order to meet this SMART objective in a future iteration of the frontend,
the Item/Entity page(s) would need to be hooked up to the code written in \lstinline{ImageUploadPage}
(debug page). There already exists UI elements in these pages for images, albeit currently
containing placeholder imagery. However, an easy way to programatically change the images
displayed and place them on a carosuel was implemented, along with an "upload image" button
which was placed at the end of the carosuel, as is the standard in other applications, although
it should be noted that the icon for this button is intersecting with the top of the parent
component on Android devices. This issue does not occur on iOS where the icon is in the center,
both horizontally and vertically, of the parent component.

\subsection{SMART Objective 3}

\textit{To produce a solution that features a fully searchable catalogue.}

\noindent \\ \textbf{SMART Objective Met?:} No

\noindent \\ This SMART objective was not completed due to time constaints.
As can be gathered from previous objective summaries as well as the incomplete
implementation stage, the frontend application was never finished.
As noted in the Design section, specifically in SMART objective 4 below,
work on the frontend was done in parallel with work in the backend, taking
an iterative approach to development. This means that the feature required in order
to satisfy this SMART objective was never implemented into the backend component
as the frontend component never progressed far enough into development to have a functional
"Home" screen, where it would have made sense and would have been intuitive to include
this feature.

\noindent \\ In order to meet this SMART objective, the "Home" screen would first need to be
properly implemented. After this is done it would be a sensible time to implement searching
functionality on the backend. While searching \textbf{could} technically be accomplished
already by using queries such as \lstinline[language=graphql]|items {}| and \lstinline[language=graphql]|enitities {}| to return
all items and entities respectively, it would be bandwidth inefficient not to mention computationally
expensive to fetch all respective entries and transmit them over the network. A more sensible
approach would be to implement a mutation (or query paramater) that allowed for searching
the results. In order for this to be implemented effectively, so that it continues to perform
well if the solution was scaled up to include more items/entities, the paramater, or search key
should be passed to GORM so that it can be included in the SQL query, assuming that adequate
validation is performed before this step to avoid SQL injection.

\subsection{SMART Objective 4}

\textit{To produce a solution including a server component which stores data and
    manages access to the system.}

\noindent \\ \textbf{SMART Objective Met?: Yes}

\noindent \\ I have decided to mark this SMART objective as completed as the server component
of the solution is in a more than usable state. The server component contains all essential features
requested by the stakeholders as well as featuring unit tests with differing types of data input
such as expected data, erroneous data and invalid data. In addition, the server component
persistently stores data in a PostgreSQL database and contains a user management module
that handles user account creation, login, issuance of tokens, as well as access to
so-called "authenticated" routes, that is to say routes that require the caller to be logged in.

\subsection{SMART Objective 5}

\textit{ To produce a solution containing a user interface that can be
    accessed via a mobile device.}

\noindent \\ \textbf{SMART Objective Met?: Yes (technically)}

\noindent \\ This SMART objective has technically been met as the solution does in fact
contain a frontend component with a user interface accessible on a mobile device.
However, while the user can log in to the system using the mobile application the user
cannot do much else.

\subsection{SMART Objective 6}

\textit{To produce a solution that is capable of efficiently storing data about
    multiple copies of an item, such as multiple copies of the same book}

\noindent \\ \textbf{SMART Objective Met?: Yes}

\noindent \\ This SMART objective has been met as the solution has been
architected in such a manner as to make this possible, and the necessary
features implemented.

\subsection{SMART Objective 7}

\textit{To produce a solution capable of supporting multiple users}

\noindent \\ \textbf{SMART Objective Met?: Yes}

\noindent \\ The solution is working as designed, multiple users
are fully supported on the system with the authentication middleware working
as designed.

\subsubsection{SMART Objective 7a}

\textit{To produce a solution that can handle multiple users interacting
    with the system simultaneously.}

\noindent \\ \textbf{SMART Objective Met?: Yes}

\noindent \\ The solution is working as intended; the system has
the concept of multiple user accounts which can access the system
independently of each other.


\subsubsection{SMART Objective 7b}
% BUG: takes multiple attempts to log in

\textit{To produce a solution that supports multiple distinct user
    accounts, providing a secure authentication system to manage access}

\noindent \\ \textbf{SMART Objective Met?: Yes* (explained below)}

\noindent \\ This SMART objective has been fulfilled since the necessary
features have been implemented. In addition, an industry standard authentication system
has been designed and implemented, including a corresponding implementation
on the frontend. However, the frontend requries multiple attempts to run it's
login code correctly.


\subsection{SMART Objective 8}

\textit{To produce a solution that has the ability to "check out" items to a
    specific user, viewable from the user interface.}

\noindent \\ \textbf{SMART Objective Met?: Partially}

\noindent \\ This feature is implemented in the backend with stub, non-functionality
UI elements in the frontend. Therefore this objective has only been partially met.

%\section{Summary of implemented usability features}

\begin{comment}
1. High contrast UI
2. Dark theme
3. No audio cues
\end{comment}

%\section{Considerations}

%\subsection{Maintenance for the developed solution}
%\subsection{Limitations of the developed solution}

%\section{Future development of the solution}

\section{Conclusion}

\noindent \\ In conclusion, this exercise was a useful learning exercise to learn about the problems
that overambitious projects face, and to learn about feature creep and it's effect on a project.

\noindent \\ In addition, I learnt a great deal about how to effectively plan a project, even if I did learn
some of that through making mistakes. I am glad to have learned from this experience despite
the suboptimal outcome.

\end{document}