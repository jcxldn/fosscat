\documentclass[../main.tex]{subfiles}

\begin{document}

\noindent I will now conduct two types of testing: tests against the success requirements as
well as usability testing. For each test I will try include three types
of inputs:

\begin{enumerate}
    \item \textbf{Expected inputs} - Inputs that are correct for the given input/field.
    \item \textbf{Erroneous inputs} - Inputs that are not correct but can still be inputted
          for the given input/field. Validation should be performed on these inputs and an error
          should be returned by the application stating that the value was not what was expected.
    \item \textbf{Invalid inputs} - Inputs that are not valid for the given input/field.
          For example, passing plain text to a GraphQL mutation that expects numeric input.
          These types of inputs should be caught by the GraphQL schema (for requests issued via
          GraphQL)
\end{enumerate}

\section{Testing against the success requirements}

\noindent Testing will now be performed in order to determine if the success requirements,
detailed in \textit{Section 1.4.3}, have been fulfilled. The testing
will not be performed in the same order as the SMART objectives were defined. Instead, the
order will roughly follow the pattern in which a user would use the program ie. logging in
before attempting to fetch authenticated routes.

\subsection{SMART Objective 7}

\textit{To produce a solution capable of supporting multiple users.}

\subsubsection{Test 1}

\subparagraph{Test Description}
\subparagraph{Input data}
\subparagraph{Expected result}
\subparagraph{Actual result}

\end{document}