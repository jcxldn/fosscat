\documentclass[../../main.tex]{subfiles}

\begin{document}

\begin{comment}
\subsubsection{Introduction OLD}

\noindent\\ The first thing I decided to work on was user account creation. This would involve using GraphQL to receive data about the new user, generating the user ID, and then placing that data in a database.
After doing some research (should I talk about the different options?) I decided to use the following three libraries for my project:

\begin{outline}
    \1 \textbf{Gin}\\
    Gin is a HTTP framework for Go. This allows me to expose my GraphQL endpoint over HTTP. After conducting some research, I choose to use gin as it is by far the fastest library available at the time of writing.
    It offers performance up to 40 times faster than it's closest alternative, \textit{"Martini"}.
    Gin is a very popular framework, with over 72 thousand stars on GitHub. Using this library benefits me as it has been battle-tested by other users and includes end to end testing, making it more than suitable for my project. (wording isn't great, trying to say it has been tested already)

    \1 \textbf{Gqlgen}\\
    Gqlgen is a Go library designed to make building GraphQL servers easy and hassle-free. It is designed with a schema-first approach, meaning the developer (me) can simply define their API using the standard GraphQL Schema Definition Language. It prioritizes type safety, and has rudimentary validation support built-in. However, I will not be relying on this validation feature, I will create my own validation system that better suits my needs.

    \1 \textbf{GORM} (Go Object Relational Mapping library)\\
    GORM is a fully-features ORM library for Go. (explain what an ORM does)
\end{outline}

\end{comment}

\subsubsection{Introduction}

This is the first iteration of my system. My aim for this iteration is to create the initial backend service for the application.
This service will provide an interface for the frontend to talk to the database via an API.

\noindent \\ An API, or \textbf{A}pplication \textbf{P}rogramming \textbf{I}nterface, is a mechanism that allows two pieces of software to communicate with each other
using a predefined protocol. I will be making use of an API to allow the client application (the frontend) to request and send data to the backend application (the backend).

\noindent \\ I am writing the backend service in Go, which is a performant, statically typed high level language designed by Google.
Go is frequently used for backend development thanks to it's excellent performance and built in memory safety.
I am also going to use \underline{GraphQL} as the query language used by the frontend to interact with the backend.

\noindent \\ GraphQL is an open-source query and manipulation language designed for use in APIs. The backend will serve as an API which will interface
with my database.
I choose to use GraphQL as it is better suited for larger, more complex data sources, and supports querying for multiple different types of data at
once, unlike a REST API. GraphQL was also a technology I am interested in learning more about as I have not implemented a complex system using it before.

\noindent \\ A REST API, also known as a RESTful API, is an application programming interface that makes use of REST. REST, or \textbf{RE}presentational \textbf{S}tate \textbf{T}ransfer,
is a standard architecture that was created as a way to manage communication over a complex network, such as the Internet. RESTful APIs are most commonly implemented using
the Hypertext Transfer Protocol, or HTTP. HTTP requests can have several methods, which is used to instruct the server what action should be performed on the requested resource.

\noindent \\ GraphQL has two main operation types that I will be making use of in my system. The first operation is a \textbf{query}. This allows the caller to request
data from the GraphQL server. In addition, the caller can make use of parameters to change the result of their query. This is the GraphQL equivalent of
a HTTP \textbf{GET} request as would be used for a REST API. The other operation type is called a \textbf{mutation}. A mutation is an operation that
allows the caller to insert new data or modify existing data on the server. It is the GraphQL equivalent of a HTTP \lstinline{POST}, \lstinline{PUT},
\lstinline{PATCH} or \lstinline{DELETE} request as used in a REST API.


\subsubsection{Web Server Setup}

GraphQL, like RESTful APIs, is usually accessed over HTTP. To achieve this, I first needed to create a basic HTTP server in Go.
Go's standard library does in fact provide functions ot create a HTTP server, and that is what I used first.
To start, I setup my Go project by running \lstinline{go mod init fosscat}, which creates the dependency file, \lstinline{go.mod}.
As specified in my design document, I then installed \lstinline{gqlgen} by adding it to the project's \lstinline{tools.go}.
I then ran \lstinline{go run github.com/99designs/gqlgen init} and \lstinline{go mod tidy} to initialize the gqlgen configuration and generate
GraphQL models. Lastly, I added boilerplate code to create a webserver using Go's standard library functions and register the GraphQL route handler.

\noindent \\ After doing this, my main file, \lstinline{server.go}, looked like this:

\begin{lstlisting}[language=Go]
package main

import (
	"log"
	"net/http"
	"os"

	"github.com/99designs/gqlgen/graphql/handler"
	"github.com/99designs/gqlgen/graphql/playground"
	"github.com/jcxldn/fosscat/backend/graph"
)

const defaultPort = "8080"

func main() {
	port := os.Getenv("PORT")
	if port == "" {
		port = defaultPort
	}

	srv := handler.NewDefaultServer(graph.NewExecutableSchema(graph.Config{Resolvers: &graph.Resolver{}}))

	http.Handle("/", playground.Handler("GraphQL playground", "/query"))
	http.Handle("/query", srv)

	log.Printf("connect to http://localhost:%s/ for GraphQL playground", port)
	log.Fatal(http.ListenAndServe(":"+port, nil))
}

\end{lstlisting}

\noindent This code checks for the presence of an \underline{environment variable} called \lstinline{PORT}. If it is present that port is used by the webserver as it's listening port. If it was not specified, port 8080 is used.
I then register routes for both GraphiQL (\lstinline{"/"}) and the GraphQL server itself (\lstinline{"/query"}).

\noindent \\ GraphiQL is a webpage that ships with GraphQL servers that allows for easy testing. It provides a web-based interface to interact with the attached GraphQL server. It is similar in function to programs like Altair which I will be using later on in this iteration to test the functionality of my code.

\subsubsection{User account creation}

The first feature I decided to work on was user account creation. This would involve asking the user for an email address, name and password, before validating it and inserting it into the database.
In addition, at a later stage, validation must be performed in order to ensure that:

\begin{itemize}
    \item The user email is not already in use
    \item The generated user ID is unique and not already in use
\end{itemize}

\noindent For this early stage of development, I decided to use an SQLite database to make things easier. I hope that I can easily switch this to PostgreSQL (as specified in my design doc) later on in the development process.

\noindent I have created a GraphQL mutation called \lstinline{createUser}. When it is called, the GraphQL library calls the \lstinline{CreateUser} function, passing any input data from the query.\\ The database connection is available at "\lstinline{r.db}".\\

\noindent My first version of this function was as follows:

\begin{lstlisting}[language=Go]
    // CreateUser is the resolver for the createUser field.
    func (r *mutationResolver) CreateUser(ctx context.Context, input model.NewUser) (*model.User, error) {
        // Create the user struct
        user := structs.User{FirstName: input.FirstName, LastName: input.LastName, Email: input.Email}
        
        // Generate a user ID
        user.ID = uuid.New()
    
        // Create the database entry
        r.db.Create(&user)
    
        // Return the created user data, converting it to a GraphQL model.
        return &model.User{
            ID: user.ID.String(),
            FirstName: user.FirstName,
            LastName: user.LastName,
            Email: user.Email
        }, nil // return nil in the error field
    \end{lstlisting}

\noindent \\ The first thing I noticed after implementing this function was that it was tedious to convert back and forth between \lstinline{structs.User} (the database object) and \lstinline{model.User} (the GraphQL object). I decided to merge these into a single object. This was done with the following lines in my GraphQL library's configuration file:

\begin{lstlisting}[language=Python]
    models:
    
    [..]
    
        # Custom models
        Checkout:
          model: github.com/jcxldn/fosscat/backend/structs.Checkout
        Entity:
          model: github.com/jcxldn/fosscat/backend/structs.Entity
        Item:
          model: github.com/jcxldn/fosscat/backend/structs.Item
        User:
          model: github.com/jcxldn/fosscat/backend/structs.User
    \end{lstlisting}

This instructs the GraphQL library to use the structs I defined for the database as if they were GraphQL models.

\subparagraph{Unique ID generation}

\noindent \\\\ I decided to use UUIDs (Universal Unique Identifiers) as IDs for all of the objects in my database. (eg. Users, Items) As seen above, I initially choose to simply call \lstinline{uuid.New()} to generate a new random UUID. However, I would soon realise that it would be beneficial to perform \textbf{validation} in order to ensure that the UUIDs were actually unique, ie. that they were not being used by another object of the same type. For example, I wouldn't want two users to have the same User ID.

\noindent \\ I decided to use a \textbf{for loop} to continuously generate UUIDs to be used as a potential User ID. I then perform a database lookup to ensure that the UUID is not already in use.

This can be done with the following code:

\begin{lstlisting}[language=Go]
    func (r *mutationResolver) CreateUser(ctx context.Context, input model.NewUser) (*structs.User, error) {
    
    [..]
    
        isFreeUuid := false
        for !isFreeUuid {
            // Generate a UUID for the user id.
            user.ID = uuid.New()
            // Check that the UUID has not been used already
            // If true, it will break out of this for loop and continue.
            isFreeUuid = util.IsUuidFree[structs.User](r.db, user.ID)
        }
    
    \end{lstlisting}

\noindent In order to achieve this and reduce code duplication across different functions, \\
I created a "IsUuidFree" utility function. Here is the code:

\begin{lstlisting}[language=Go]
    
    func IsUuidFree[T any](db *gorm.DB, id uuid.UUID) bool {
        obj := new(T)
        err := db.Model(obj).Select("id").Where("id == ?", id.String()).First(&obj).Error
        if errors.Is(err, gorm.ErrRecordNotFound) {
            // Record not found, so user id is free
            return true
        } else {
            // Record was returned successfully, therefore the user exists
            return false
        }
    }
    
    \end{lstlisting}

JAMES: this initially was different but I changed it to make it simpler. Should I include the old version as well?\\

\noindent This function makes use of \textbf{generics}. As per the Go docs:

\begin{quotation}
    \textit{
        With generics, you can declare and use functions or types that are written to work with any of a set of types provided by calling code.
    }
\end{quotation}

\noindent To simplify, generics mean that I can pass any struct (\textbf{T}) to the function.
For example, if I call the function with:\\

\begin{lstlisting}[language=Go]
        util.IsUuidFree[structs.User](r.db, user.ID)
    \end{lstlisting}

\noindent \\Then T is set to the type \lstinline{structs.User}.\\

\noindent \textbf{GORM} (my database library) works by defining a struct to query for which corresponds to a table in the database (in this case the \lstinline{Users} struct corresponds to the \lstinline{users} table). We can then perform SQL actions on this table, such as Select. \\

\noindent Therefore, the GORM db call from above:

\begin{lstlisting}[language=Go]
        db.Model(obj).Select("id").Where("id = ?", id.String()).First(&obj)
    \end{lstlisting}

\noindent is the equivalent of:

\begin{lstlisting}[language=SQL]
        SELECT id FROM users WHERE id == ? LIMIT 1 VALUES ("value of id.String()")
    \end{lstlisting}

\subparagraph{Testing the user account creation flow}

\noindent \\ Now that I have implemented user account creation, I need to test it to verify that my solution works as expected. I have added a GraphQL query to list all users, which I will use in conjunction with the \lstinline{createUser} mutation.

\noindent \\ I am using a piece of software called \textbf{Altair}, which is an interactive way to make GraphQL queries. I start by creating my GraphQL query which includes the \lstinline{createUser} mutation, making sure to set all required fields:\\

\includegraphics[width=15cm]{implementation/first iteration/createUser altair.png}

\noindent \\ Running the query results in the following response:\\

\includegraphics[width=15cm]{implementation/first iteration/createUser altair ran.png}

\noindent \\ Let's check the database for our new user to ensure it was added successfully:\\

\includegraphics[width=15cm]{implementation/first iteration/createUser query db.png}

\noindent \\ We can see that the user is created successfully, added to the database, and the specified fields (line 12 onwards in the query) are returned to the client.

\subsubsection{Problems encountered when moving to PostgreSQL}

\noindent At this stage, with most of the core functionality implemented, I decided to switch back to PostgreSQL. However, when doing this I encountered two problems:

\subparagraph{Problem 1 - Entity relation errors}

\noindent \\ When creating the database in PostgreSQL, I encountered the following error, displayed in the backend logs.

\begin{lstlisting}[language=SQL]
    /backend/database.go:35 ERROR: relation "entities" does not exist (SQLSTATE 42P01)
    
    [17.308ms] [rows:0] CREATE TABLE "checkouts" ("id" text,"created_at" timestamptz,"updated_at" timestamptz,"deleted_at" timestamptz,"take_date" timestamptz,"return_date" timestamptz,PRIMARY KEY ("id"),CONSTRAINT "fk_entities_checkouts" FOREIGN KEY ("id") REFERENCES "entities"("id"))
    \end{lstlisting}

\noindent \\ This error could be traced to the following line in my code, where the database is "migrated" by GORM, which means that it attempts to create the necessary tables and columns to match the structs I have defined.

\begin{lstlisting}[language=Go]
        // "Migrate" the schema
        // This will create tables, keys, columns, etc.
        // See https://gorm.io/docs/migration.html
        // Note that we need to pass each struct in our schema.
        db.AutoMigrate(&structs.Checkout{}, &structs.Entity{}, &structs.Item{}, &structs.User{})
    \end{lstlisting}

\noindent \\Could do: For example struct A creates this table? show it off?

\noindent \\ This error prevented me from progressing with the backend implementation, as the program would error out during table creation (should I remove this line?)

\noindent \\ After some investigation, I found out that this error occurs when tables that have dependencies on each other are created at the same time (in the same \lstinline{AutoMigrate} call). The fix was to create the dependent table first followed by the table that depended on it, the code for which can be seen below:

\begin{lstlisting}[language=Go]
        // "Migrate" the schema
        // [..]
        db.AutoMigrate(&structs.Checkout{})
        // Item has a dependency on Entity, so do them in the correct order
        // to avoid "relation does not exist" error during table creation.
        db.AutoMigrate(&structs.Entity{})
        db.AutoMigrate(&structs.Item{})
        db.AutoMigrate(&structs.User{})
    \end{lstlisting}

\noindent \\ After making this change, I decided to validate and test it before moving on.

\noindent \\ I decided to test this change by first deleting all the database tables and then starting the backend, which should create (or "migrate") all of the tables.

\noindent \\ Firstly I will connect to the database and delete the tables. I have attached the console output and have annotated what I am doing to make it easier to understand.

% Define psql commented language for lstlistings
\lstdefinelanguage{custom_psql_commented}{
    alsoletter={\=,\#,\-},
    keywords={fosscat\=\#, fosscat\-\#},
    morecomment=[l]{//}
}

\begin{lstlisting}[language=custom_psql_commented]
    // Run 'psql' to connect to the database
    [james@linux cs-coursework]$ psql -h localhost -U fosscat -W
    // Enter the password (it is not displayed)
    Password: 
    psql (15.4, server 16.1 (Debian 16.1-1.pgdg120+1))
    WARNING: psql major version 15, server major version 16.
             Some psql features might not work.
    Type "help" for help.
    
    // List all tables, we can see that they exist
    fosscat=# \dt
              List of relations
     Schema |   Name    | Type  |  Owner  
    --------+-----------+-------+---------
     public | checkouts | table | fosscat
     public | entities  | table | fosscat
     public | items     | table | fosscat
     public | users     | table | fosscat
    (4 rows)
    
    // Delete the schema containing all of the tables
    fosscat=# DROP SCHEMA public CASCADE;
    NOTICE:  drop cascades to 4 other objects
    DETAIL:  drop cascades to table users
    drop cascades to table checkouts
    drop cascades to table entities
    drop cascades to table items
    DROP SCHEMA
    // Re-create the schema
    fosscat=# CREATE SCHEMA public;
    CREATE SCHEMA
    // Set default permissions on schema
    fosscat=# GRANT ALL ON SCHEMA public TO public;
    GRANT
    // List all tables, we can see that there aren't any
    fosscat=# \dt
    Did not find any relations.
    // Quit
    fosscat-# \q
    \end{lstlisting}

\noindent As can be seen above, the database now contains no tables. Next, let's start the backend, where GORM should recreate the database tables. Below is the startup log:

\begin{lstlisting}
    [james@linux cs-coursework]$ GIN_MODE=release ./start-backend.sh
    2023/12/05 14:23:08 [database] connected. migrating...
    2023/12/05 14:23:08 [database] migrated, done.
    2023/12/05 14:23:08 [resolver] db not set, setting.
    \end{lstlisting}

\noindent You can see that no errors were produced in the console, indicating that all of the necessary tables and columns were created successfully. To check this, let's connect to the database again and list the tables:

\begin{lstlisting}[language=custom_psql_commented]
    // Run 'psql' to connect to the database
    [james@linux cs-coursework]$ psql -h localhost -U fosscat -W
    // Enter the password (it is not displayed)
    Password: 
    psql (15.4, server 16.1 (Debian 16.1-1.pgdg120+1))
    WARNING: psql major version 15, server major version 16.
             Some psql features might not work.
    Type "help" for help.
    
    // List all tables, we can see that they exist
    fosscat=# \dt
              List of relations
     Schema |   Name    | Type  |  Owner  
    --------+-----------+-------+---------
     public | checkouts | table | fosscat
     public | entities  | table | fosscat
     public | items     | table | fosscat
     public | users     | table | fosscat
    (4 rows)
    
    // Quit
    fosscat=# \q
    \end{lstlisting}

\noindent We can see that the tables were created successfully.

\subparagraph{Problem 2 - Operator does not exist}

\noindent \\ However, when attempting to create a new user in the database I encountered another error:

\begin{lstlisting}[language=SQL]
    ./backend/util/user.go:14 ERROR: operator does not exist: text == unknown (SQLSTATE 42883)
    
    [0.369ms] [rows:0] SELECT "id" FROM "users" WHERE id == '8596a222-930e-42f0-841e-9d95993668a4' AND "users"."deleted_at" IS NULL ORDER BY "users"."id" LIMIT 1
    \end{lstlisting}

\noindent \\ This error states that it cannot compare \lstinline{id} with the given UUID (\lstinline{'8596a222-930e-42f0-841e-9d95993668a4'}) because the operator \lstinline{==} does not exist.
This error points to my \lstinline{IsUuidFree} function, which I talked about in the \textbf{User account creation} section of this iteration.

\noindent \\ After some research, I found that the error occurs because the operator \lstinline{==} does not exist in PostgreSQL. \lstinline{==} is commonly used in programming languages to perform a deep comparison of two objects, and I assumed that the same would be true for PostgreSQL.
Interestingly, the issue did not manifest itself until after the switch to PostgreSQL, meaning that SQLite handles \lstinline{==} as I expected.
The fix was simple; Change \lstinline{'=='} to \lstinline{'='}.

\noindent \\ This meant that the line

\begin{lstlisting}[language=Go]
        err := db.Model(obj).Select("id").Where("id == ?", id.String()).First(&obj).Error
    \end{lstlisting}

\noindent became

\begin{lstlisting}[language=Go]
        err := db.Model(obj).Select("id").Where("id = ?", id.String()).First(&obj).Error
    \end{lstlisting}

\begin{comment}
POLISH END
\end{comment}

\subparagraph{Problem 3 - "... violates foreign key constraint"}

\noindent \\ When trying to create a new user, the following error message would appear:

\begin{lstlisting}[language=SQL]
    /backend/graph/resolver/mutation.go:90 ERROR: insert or update on table "users" violates foreign key constraint "fk_checkouts_user" (SQLSTATE 23503)
    [2.691ms] [rows:0] INSERT INTO "users" ("id", "created_at", "updated_at", "deleted_at", "first_name", "last_name", "email", "hash") VALUES [..]
    \end{lstlisting}

\noindent This error occurred when trying to create a user. Upon reading into it the error occurred because of how I defined my foreign keys for GORM. For example, I had the following struct definition for \lstinline{Checkout}:

\begin{lstlisting}[language=Go]
    type Checkout struct {
        gorm.Model
        ID         uuid.UUID
        User       User `gorm:"foreignKey:ID"`
        TakeDate   time.Time
        ReturnDate time.Time
    }
    \end{lstlisting}

\noindent Upon reading the GORM documentation, I realised this should actually be:

\begin{lstlisting}[language=Go]
    // Checkout belongs to a User, User.ID (UserID) is the foreign key
    type Checkout struct {
        gorm.Model
        ID         uuid.UUID
        User       User
        UserID     uuid.UUID
        TakeDate   time.Time
        ReturnDate time.Time
    }
    \end{lstlisting}

\noindent After applying this change, the program worked as expected. I tested everything by attempting to create a new user in Altair:\\

\includegraphics[width=15cm]{implementation/first iteration/postgreproblems altair working.png}

\noindent \\ This query returned successfully and the user was created without any errors.

\subsubsection{Adding foreign keys for lists}

\subsubsection{Issues with nested queries (queries using multiple tables)}

\subsubsection{Adding the remaining queries}

CreateCheckout validation

\subsubsection{Testing}

\subparagraph{Test Plan}

\noindent \\ My plan for testing this iteration was to create \underline{unit tests} for my project. Unit tests are an automated set of tests designed to ensure that the tested application works correctly.

\noindent \\ After some research, I decided to use the \underline{testify} testing framework, which is a toolkit with assertions and mocking support that works well with standard go functions.
I also decided that it would be beneficial to have a testing database, so I decided to create a script that spins up an ephemeral (short-lasting) database container using the \textit{testcontainers} package. This database will only run for the duration of the test suite.

\noindent \\ A snippet of this script is included below: (should I just remove this from the writeup? not *really* needed)

\begin{lstlisting}[language=Go]
// excerpt of backend/test/common/database.go

// SetupSuite is called before any tests run
func (s *DatabaseTestSuite) SetupSuite() {
	s.dbCtx = context.Background()
	// Create a container request for a container running the "postgres" image
	req := testcontainers.ContainerRequest{
		Image:        "postgres",                                                                      // container image to use
		ExposedPorts: []string{"5432"},                                                                // ports to expose to host
		WaitingFor:   wait.ForLog("database system is ready to accept connections").WithOccurrence(2), // trigger to define when container has started up
		Env: map[string]string{ // environment variables
			"POSTGRES_DB":       "fosscat",
			"POSTGRES_USER":     "fosscat",
			"POSTGRES_PASSWORD": "fosscat",
		},
	}

	// Note that we do not define any persistent storage so the database will start from scratch every time it is created.

	// Create the container **and** wait for it to start up
	dbContainer, err := testcontainers.GenericContainer(s.dbCtx, testcontainers.GenericContainerRequest{
		ContainerRequest: req,  // specify the container request
		Started:          true, // automatically start once created
	})

	s.dbContainer = dbContainer

	if err != nil {
		panic(err)
	}

	// Determine the IP address of the container
	ctrIp, _ := dbContainer.ContainerIP(s.dbCtx)

	// Log that the database is now available
	log.Default().Printf("[test/common/database]: ephemeral db available on %s:5432", ctrIp)

	// Define connection details for the database
	dsn := fmt.Sprintf(
		"host=%s user=fosscat password=fosscat dbname=fosscat port=5432", ctrIp,
	)

	// Attempt to connect to the db
	db, err2 := gorm.Open(postgres.Open(dsn), &gorm.Config{})

	if err2 != nil {
		panic(err)
	}

	// Call migrate function (defined in backend/database/database.go) to create tables
	database.Migrate(db)

	// make the GORM instance available to tests
	s.DB = db

}
\end{lstlisting}

\noindent After setting this up, I create TestSuites that could be inherited from in order to reduce code duplication. I first created the following suites:

\begin{outline}
    \1 \lstinline{DatabaseTestSuite} - A test suite that provides the ephemeral database as seen above
    \2 \lstinline{UserDatabaseTestSuite} - Handles creating user accounts for child test suites to use, includes assertions and it's own user tests.
    \2 \lstinline{EntityDatabaseTestSuite} - Handles creating entities for child test suites use, includes assertions and it's own entity tests.
\end{outline}

\noindent Now the ground-work was in place, it was time to write the unit tests themselves.

\noindent I started with User and Entity tests first, placing them in their respective suites.
I used a \textbf{code coverage} tool to ensure that every line of code was covered.
This means that the complete functionality of a function or line of code was tested in my tests.
For example, with an if statement, both outcomes must be tested for.

\noindent \\ When creating tests, it was important to ensure that all combinations of inputs were tested for.
For example, for \textbf{Checkouts} I created tests for:

// INCLUDE EXAMPLES OF TESTS

\begin{outline}
    \1 Checkout creation
    \2 With all fields persistent
    \2 Without a return date
    \2 Without a take date
    \2 With only the user ID field
    \1 Checkout deletion (TBD)
    \2 Deleting a non-existent checkout
    \2 Deleting an existing checkout
\end{outline}

\noindent This allowed me to ensure that all parts of my program were tested properly.

\noindent TODO. Rewrite this I am repeating myself quite a bit.


\subparagraph{Test Results}

\noindent \\ My unit testing uncovered \underline{four problems}:

\begin{outline}
    \1 (User creation): Email validation is broken
    \1 (User creation): Password can only be 72 characters
    \1 (Checkout creation): Take and Return dates not being stored in the database
    \1 (Item creation): Item title not being stored in the database
\end{outline}



\paragraph{Errors encountered during testing}

\subparagraph{Issues with user creation}

\noindent \\ I am using bcrypt for password hashing. When testing user creation with long passwords, I noticed that tests would fail with errors when the password was greater than 72 bytes (characters).
Some further research revealed that this is a limitation of bcrypt itself.
To avoid the unintentional behaviour caused by these errors, I added validation for the password length.
If the validation is not successful, the user will be presented with a descriptive error and the user will not be added to the database.

\noindent \\ After fixing this issue, I noticed that whilst writing tests for invalid e-mail addresses that the email address always appeared to be valid, even when it shouldn't.
I am using the go \lstinline{emailVerifier} library to validate email strings passed to it, and it turned out I was calling the library incorrectly.
After reading the docs for the library, I elected to use the much simpler \lstinline{ParseAddress} function instead of the more complicated \lstinline{VerifyEmail} function.
The VerifyEmail function attempts to connect to the specified domain in the email address string to confirm that it can receive emails, which is overkill for my use.
Instead I will be using the ParseAddress function which merely applies a regular expression to ensure that the email has the correct syntax.

\subparagraph{Issues with checkout creation}

\textbf{// INCLUDE DB SCREENSHOTS?}

\noindent \\ When creating tests for checkouts, I noticed whilst using a database viewer that I was missing data in the database after creating a checkout. Specifically, the \lstinline{TakeDate} and \lstinline{ReturnDate} fields would always be empty.
Upon taking a closer look at my code, it was soon obvious why this was the case.

\begin{lstlisting}[language=Go]
// Truncated from the original

func CreateCheckout(db *gorm.DB, input model.NewCheckout) (*structs.Checkout, error) {
	// Create a Checkout struct
	checkout := structs.Checkout{}

    // Set checkout.User to match the user object we queried from the database
    checkout.User = *user

	// Create the database entry
	db.Create(&checkout)

    // Return the entry and no error (nil)
	return &checkout, nil
}
\end{lstlisting}

\noindent As you can see, \lstinline{TakeDate} and \lstinline{ReturnDate} were never set in the \lstinline{checkout} struct, so when we called \lstinline{db.Create(&checkout)}, these values were \lstinline{nil}.
However, the fix was not as simple as setting \lstinline{TakeDate} and \lstinline{ReturnDate}, as these fields are optional during creation. As the user cannot be sure about either the take or return dates when creating a checkout, they are marked as optional so that they do not have to be specified during checkout creation, instead allowing the user to edit the record and add them at a later date.

\subparagraph{Issues with item creation}

\noindent \\ Similar to before, when creating tests for my items, I noticed in the database viewer that the Title field would always be empty, even if it was passed to my CreateItem function.
Similarly to before, I had forgotten to set the item name in the struct before adding the struct to the database.
This was a simple fix, where I only needed to add three lines:

\begin{lstlisting}[language=Go]
func CreateItem(db *gorm.DB, input model.NewItem) (*structs.Item, error) {
	// Create a Item struct
	item := structs.Item{}

    // [!] I added these three lines to set the title
    if input.Title != nil {
		item.Title = *input.Title
	}

    // [truncated]
}
\end{lstlisting}

\begin{comment}
\noindent \\ commits:
\begin{outline}
    \1 User/create: email validation was broken, 72 char limit for passwords
    \2 https://github.com/jcxldn/fosscat/commit/605df88838143f717779ba289b58d567b5929a5a
    \1 Checkout/create: take and return dates not being stored in database
    \2 https://github.com/jcxldn/fosscat/commit/64d53884117ffc035b523457f9ee87433868e65c
    \1 Item/create: title not stored in DB
    \2 https://github.com/jcxldn/fosscat/commit/73aa435cc089d95f25d88ebdff6b77fc95cd3543
\end{outline}
\end{comment}

\subparagraph{Summary}

\noindent \\ After fixing these errors, I was able to create the rest of my tests.

\noindent \\ As my code is hosted on GitHub, I setup their Continuous Integration (CI) service to automatically run my test suite every time I committed to the repository. After setting this up (not going to detail here as is not relevant?), I was able to queue the test and was greeted with the following result: \\\\

\includegraphics{implementation/first iteration/test-dashboard}

\noindent \\ Here is an excerpt from one of the test runs, showing the output from one of the test suites:

\begin{lstlisting}
=== RUN   TestUserTestSuite
=== RUN   TestUserTestSuite/TestCreateUser
=== RUN   TestUserTestSuite/TestCreateUserInvalidEmail
=== RUN   TestUserTestSuite/TestCreateUserInvalidPasswordTooLong
=== RUN   TestUserTestSuite/TestGetAllUsers
--- PASS: TestUserTestSuite (6.21s)
    --- PASS: TestUserTestSuite/TestCreateUser (0.91s)
    --- PASS: TestUserTestSuite/TestCreateUserInvalidEmail (0.00s)
    --- PASS: TestUserTestSuite/TestCreateUserInvalidPasswordTooLong (0.00s)
    --- PASS: TestUserTestSuite/TestGetAllUsers (0.00s)
\end{lstlisting}

\noindent \\ As we can see both from the test summary generated by GitHub and the output of the test runs themselves, all of the tests are passing correctly.

// show code coverage?

\subparagraph{Fixing error XYZ}

\subsubsection{Evaluation}

\noindent In this iteration, I have successfully written code that can create a database, create the necessary tables to store data, and exposes functions to create and store data in the database.

\noindent \\ Using GraphQL, user accounts can be created, where a unused unique identifier is found and used, as well as the creation of a salted password hash to securely store the user's password. In addition, checkouts, entities and items can be created each with their unique identifiers. Where necessary, such as with an checkout, we can use relationships to tie the checkout to a specified user object residing in the database.

\noindent \\ The next iteration will focus on adding security to the backend. For example, I will need to add authenticated routes, where users can only view the contents of the route if they are logged in, as it will contain sensitive data.

\end{document}