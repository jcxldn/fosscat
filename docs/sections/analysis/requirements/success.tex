\documentclass[../../../main.tex]{subfiles}

\begin{document}

\noindent The overall objectives for the system.

\noindent \\ To measure the overall effectiveness of the system, targets must be set before writing the program.
These targets will help in the evaluation stage to determine wheather our objectives have been met.
These objectives will be \textbf{SMART}, i.e:

\begin{outline}
      \1 \textbf{Specific}\\
      What objective needs to be accomplished?
      \1 \textbf{Measurable}\\
      How can we quantify this objective?\\
      How will the success of this objective be measured? (quantitatively or qualitatively)
      \1 \textbf{Achievable}\\
      Is this objective achievable and realistic? If so, how to you plan to achieve them?
      \1 \textbf{Relevant}\\
      How does this objective benefit the end-users of this application as a whole?\\
      Why has this goal been set?
      \1 \textbf{Timely}\\
      Can this objective be completed within an appropriate time frame?\\
      At what stage in the software development lifecycle will you start implementing this goal?\\
      In which order will any sub-objectives be completed?
\end{outline}

\paragraph{The Project's SMART Objectives}

\begin{enumerate}
      \item \textbf{To produce a solution capable of persistently cataloguing library books and other assets for a school.}\\
            Having a workable system is a requirement in order for the system to be used effectively.
            To achieve this objective, a database will be used.
            I will evaluate my success in completing this objective by conducting a test where a user of the system is
            tasked to input a small set of data from their environment. (for example, input 5 library books into the system)
            The user will then be asked to close and re-open the application, before trying to fetch the inputted data.
            This will test the core functionality of the system. This objective is relevant as it the key feature of the system.
            This objective, being the core objective of the system, will be continually worked on throughout the development stage.

      \item \textbf{To produce an intuitive and easy to use solution}\\
            I will evaluate my success on this objective by having a new user without any prior training or advice use the system and
            try to carry out a number of tasks without any assistance. If the user is able to successfully complete the tasks
            I will consider the system to be intuitive and easy to use and therefore this objective satisfied.
            To achieve this I will design my system to have a consistent layout based on \textbf{Material Design 3}, (\url{https://m3.material.io/})
            the design language used by Google products and many apps running on the Android operating system.
            I will also use language that is:
            \begin{enumerate}
                  \item appropriate for the situation the product will be deployed in\\(a school setting with young children who will be using the system)
                  \item easy to understand (so that children can interact with the system)
            \end{enumerate}
            I will also use meaningful error messages so that the user has a clear understanding of the problem that has occurred.
            This objective will benefit the end user as a intuitive and easy to use solution is critical to the usefullness of the project. If the end product is not easy to use, it is less likey to be used and accepted by my stakeholders.
            This objective will be worked on during the development process, and so will be completed by the time development concludes. I will mock-up a version of the user interface in the design stage and will continuously iterate on the user interface during development.\\\\
            To support the implementation of this objective, the following sub-objective has been added. This sub-objective was suggested by a student stakeholder during the previous interview stage.
            \begin{enumerate}
                  \item \textbf{To produce a solution that supports uploading thumbnail images for items stored in the system.}\\
                        The solution should support the uploading of small images from the user interface to be used as a thumbnail for an item entry.
                        If an image is available, it should be viewable from the user interface to serve as an aid to the user.
                        In order to determine if this objective has been fulfilled, a user will be instructed to upload a photo to an existing item in the catalog.
                        The objective will be fulfilled if the image uploads sucessfully and subsequently appears in the user interface.
                        This sub-objective will likely require several algorithms to validate and process the image, which, to ease development, will be planned in the design stage.
                        This sub-objective will be worked on throughout the development process, but particuarly during the development
                        of objectives 4 and 5 since additional features will need to be implemented to fulfill this objective.
            \end{enumerate}

      \item \textbf{To produce a solution that features a fully searchable catalogue}\\
            In order to implement the list of features compiled in \lstinline{Section 1.3.2}, the solution's entries
            must be fully searchable. This means that a user can input a query into a search box in the user interface,
            such as a partial name of an catalogued item, and the system should return a list of matching items.
            The success of this objective can be measured by having a user search for an item from the user interface.
            For this objective to be fulfilled, the user should search for a item stored in the system by entering,
            for example, part of the item's name, and the system should return said item in a list of possible
            matches. If the system does not return the item, this objective would not have been fulfilled.

      \item \textbf{To produce a solution including a server component which stores data and manages access to the system}\\
            Fulfilling this objective will the first priority, as the "front-end" component detailed below cannot be completed
            until this objective is fulfilled. This objective will be initially implemented at the start of the development
            stage, (eg. the first iteration of the system) before work begins on the user interface. This objective will then
            be worked on in parallel with the below objective in order to make use of an iterative development methodology.

      \item \textbf{To produce a solution containing a user interface that can be accessed via a mobile device}\\
            The system's user interface should be able to be accessed using a mobile device, such as a tablet
            or mobile phone. This objective can be measured by attempting to open the user interface on such a device.
            If the user interface loads sucessfully, this objective has been fulfilled. In order to achieve this objective,
            certain design decisions have been made. The "front-end", or client application that hosts the user interface
            will be made using React Native and Expo, a set of frameworks for creating user interfaces specifically
            tailored for use on mobile devices. The stakeholders for this system have requested that the user interface
            be accessible on a mobile device as it is what the end-users of the system have available to them.
            Therefore, this objective will greatly benefit the end-users of the system by allowing them to access the interface
            without needing additional equipment. Development of this objective will commence following initial
            iterations of the server component, detailed above.

            %\item \textbf{To produce a solution that features reporting for overdue and/or lost books}
            %\item \textbf{To produce a solution that includes a curated "suggested reading list" for each borrower}

      \item \textbf{To produce a solution that is capable of efficiently storing data about multiple copies of an item, such as multiple copies of the same book.}\\
            The system should be capable of storing common data in an efficient manner. For example, if multiple copies of the same book
            are inputted into the system, common data such as the book's author and title should only be stored once.
            The system should be laid out in such a manner to make this possible. For this objective to be fulfilled,
            the system should have the concept of both "items" and copies (instances) of said item. The item datatype
            include data such as eg. a book title, whereas the instance datatype should include data such as if
            the item has been checked out, and to who.\\\\
            I plan to achieve this objective by splitting item data into
            two distinct structures. To make this objective easier to fulfill, I will plan this approach during
            the design stage, before development commences. This objective will benefit end-users as, when adding multiple
            copies of an item, (such as multiple copies of a book) data such as the item title will only need inputting once.
            This reduces the time required to input such data into the system as well as reducing the chance of entering incorrect data
            as this data will not have to be entered multiple times. This has the effect of supporting objective 3 (fully searchable catalogue)
            as it reduces the chance of item data such as book titles differing between different entries in the database.

      \item \textbf{To produce a solution capable of supporting multiple users}\\
            The system should be capable of supporting multiple different users accessing the system at the
            same time. In order to fulfill this objective, the following two sub-objectives must be met:
            \begin{enumerate}
                  \item \textbf{To produce a solution that can handle multiple users interacting with the system simultaneously}\\
                        The system should be able to support multiple users connecting to and interacting with the system at the
                        same time without error. The success of this objective will be measured by having multiple users launch
                        the user interface simultaneously and perform various actions on the system. The objective will have been
                        fulfilled if the system handles these requests adequately without raising an error.
                        This objective benefits the end-users of the system by making the system more robust and useful.
                        This objective is especially important as the stakeholders have indicated that the system should be future-proofed,
                        that is to say it should be able to cope with their upcoming expansion next acedemic year.
                        In order to support this sub-objective, the following sub-objective will also need to be met.
                        These sub-objectives will be completed in parallel during the development of the 3rd objective (server component).
                  \item \textbf{To produce a solution that supports multiple distinct user accounts, providing a secure authentication system to manage access}\\
                        The system should be architected in such a way that it can distinguish which user is
                        performing an action. To fulfill this objective, the user should adequately handle the creation
                        of multiple user accounts without raising an error. The system should allow for secure access
                        through the use of a user identifier (email/username) and password. In order for the system to be
                        secure and to make it more robust, the system should require users to enter the corresponding password
                        for their user account, which is validated before access to the system is granted.
                        This objective will require a "user account login" algorithm to be produced. In order
                        to streamline the development process, this algorithm will be planned in the design section.
                        This objective is beneficial to the users of the system as it affords privacy and enforces accountability
                        for the users of the system. This objective also makes the system more robust, bringing the system's capabilities
                        in line with the previously researched alternative systems.
            \end{enumerate}

      \item \textbf{To produce a solution that has the ability to "check out" items to a specific user, viewable from the user interface.}\\
            In order for this objective to be fulfilled, the solution should have the ability to "check out" items
            to specific users. The success of this objective will be measured by attempting to check out an item
            to a user and evaluating weather the action is reflected in the user interface.
            This objective is relatively easy to implement; the required functionality will be developed during the
            development of objectives 3 and 4. This objective benefits the end-users of the system as it directly addresses
            one of the stakeholder's requirements for the system, directly contributing to the system's usefullness.

\end{enumerate}

\begin{comment}
Tutor:
add quantifiable criteria for this objective

example quantifiable stuff:
large buttons
ability to to xyz (easily?)
launch the program
how will it work

stakeholder requirements may overlap with success requiremetns

GS:
Produce a system that manages inventory in a statistical and written/informative format
having the information automatically produce a graph
to clearly show when stock is low

to have a supplementary phone app that can be downloaded with the provision to
scan and check in/out items, stored in a database.

quantiative manner (numbers wise)
can update a spreadsheet automatically

able to be used remotely (T in SMART)

data inputted to the system will be processed within a time period to produce a usable outcome
(techy stuff goes here)

make use of qr scanning library to easily scan QR codes that will be placed on

display specific information about different items depending on their type
- number of inventory available
- highlight stock that is low

predict how much you are spending on consumables
eg. budgeting as an objective us ea library

asset value figure for budgeting
are you within the budget or not
\end{comment}

\end{document}