\documentclass[../../../main.tex]{subfiles}

\begin{document}

\begin{comment}
based on research etc
select the features from the research that will be incorporated
and explain what they do
from sortly, steal x feature because y
include things you won't include as well (out of scope), because xyz
\end{comment}

\noindent I have compiled a list of several features. Some of these features were requested
by stakeholders, whilst others were observed in comparable solutions already
on the market.

\noindent Taking note of my research as well as feedback from stakeholders,
there are a number of features that could be implemented that would be applicable
to my solution.

\paragraph{Features to be incorporated into my solution}

\begin{enumerate}
    \item \textbf{The ability to represent multiple copies (instances) of the same item, without duplicating shared data.}\\
          This would allow multiple copies of a book to exist in the database, with shared data (such as book title, author, etc), only being stored once.
          This reduces the amount of time needed to add items to the database, improving efficiency and saving time for the user.
          In addition, there is a smaller chance of mis-entering data and causing inconsistencies.
          This would require the following sub-features:
          \begin{enumerate}
              \item \textbf{Storage of items as separate "item" and "entity" objects in a database}\\
                    \textit{Shared data would be stored in the "item" object (eg. details about
                        a specific book), with copy-specific data being stored an "entity" object,
                        repesenting a certain copy of an item. (eg. a book)
                        This reduces the time to insert multiple copies of the same item into the
                        system, saving time and increasing efficiency}
              \item \textbf{The ability to store details about an "entity" object's corresponding "item" object.}\\
                    \textit{This is required in order to allow the system to fetch the
                        corresponding item for the given entity.}
          \end{enumerate}

    \item \textbf{The ability for multiple users to interact with the system}\\
          In addition, following best practices as well as to ensure feature parity with the
          previously-researched solutions, I have chosen to expand this feature by adding:
          \begin{enumerate}
              \item \textbf{The ability for users to log-in to individual user accounts}
              \item \textbf{The ability for users to securely access the system}
          \end{enumerate}
          The specifics of this will be planned in the \underline{Design} section.

    \item \textbf{The ability for the system to be accessed from anywhere over the Internet}\\
          \textit{Dependent on: Requirement 2(a), 2(b)}\\
          Allowing users to access the system from anywhere improves it's usefulness and ensures that access is always available (how to improve?)

    \item \textbf{The ability for the system to be accessed from a mobile device}\\
          This allows the system to be  easily accessed by prospective users.\\
          \begin{enumerate}
              \item \textbf{... from anywhere, not just on-site}\\
                    \textit{Dependent on: Requirement 3}
          \end{enumerate}


          Allows students (issued with mobile devices only) to access the system.
          \begin{enumerate}
              \item ...in order to check details of books and perform other actions whilst they are at home, outside of school hours.
          \end{enumerate}

    \item \textbf{The ability to search through objects stored in the database}\\
          This allows users to quickly access database objects in a intuitive and user-friendly manner.

          \begin{enumerate}
              \item \textbf{The ability for the program to auto-complete with suggestions when typing a partial search query}\\
                    This is a relatively simple to implement yet remarkably powerful time-saving tool for the end user.
                    This has the effect of making the program easier to use as well being a time-saving mechanism for the user.
          \end{enumerate}
\end{enumerate}

\paragraph{Features that will not be incorporated into my solution}

\noindent \\ Due to time constraints, certain features deemed non-critical (not requested by the stakeholders)
have been deemed out of scope and will not be implemented into my solution at this time.

\begin{enumerate}
    \item \textbf{The ability for a user to exist with "administrator" permissions who can perform management actions on other users}
    \item \textbf{The ability to create an "audit log" of what user account performed which action}
    \item \textbf{For books, the ability for the program to suggest similar titles to the currently selected title}
\end{enumerate}

\noindent \\ However, since my solution will be documented both through the use of the
development stage and through extensive code comments in the source I am hopeful that these
features can be implemented in the future to bring feature-parity with the existing solutions
researched previously.

\end{document}