\documentclass[../../../main.tex]{subfiles}

\begin{document}

\paragraph{Interview with Mrs. Young, staff member in charge of the library}

\subparagraph{What would you consider your skill level to be regarding technology?}

\noindent \\ I would descibe my skill level as moderate, I can get by and do what I need to.

\subparagraph{Do you currently have a way to manage inventory?}

\noindent \\ We have a paper card system that is filled in by hand showing who has borrowed a book and what date it is due back.

\subparagraph{What aspects of this solution do you like?}

\noindent \\ It works!

\subparagraph{What aspects of this solution do you dislike?}

\noindent \\ It is time consuming. Writing everything out by hand takes a lot of time and we have to be around do it.
It would be better if the children could check books out themselves.

\subparagraph{What features would you require in a custom solution?}

\noindent \\ To instantly be able to see which books were checked out, perhaps on the main page of the application, the ability to search for when an item was taken out, see who has it and when it is due back.

\subparagraph{What features would enhance your experience?}

\noindent \\ The ability to search for books by their attributes. For example, the ability to search for books by author or category.
I would like to be able to search for books using eg. the book's name instead of having to remember a book "code". It should be intuitive to search for any book I want to get details for.

\paragraph{Interview with Ella, head student librarian}

\subparagraph{What would you consider our skill level to be regarding technology?}

\noindent \\ I would say that I am quite good at technology! We are given iPads that we use to learn and I know how to use that.

\subparagraph{Do you currently have a way to manage inventory?}

\noindent \\ I think so, but I'm told it's tedious and we aren't allowed to check in/out books ourselves. I wish that we could, it would make it easier not having to wait for a teacher to be around!

\subparagraph{What aspects of this solution do you like?}

\noindent \\ I do not like our current solution very much.

\subparagraph{What aspects of this solution do you dislike?}

\noindent \\ Even as a "head" librarian I cannot do much on my own, it is annoying having to wait all of the time for a teacher to be around to check out books.
Also, we can't see what books are checked out by ourselves. We have to ask a teacher to look through the papers!

\subparagraph{What features would you require in a custom solution?}

\noindent \\ I would like to have an app on my iPad! I feel like that would be very modern instead of using old-fashioned paper.

\subparagraph{What features would enhance your experience?}

\noindent \\ If I was able to "reserve" a checked out book that would be great! I hate it when a book I want to read has already been taken by someone else! It would be nice to be notified when it comes back to the library.
I would also like to be able to extend the due date, for times when I do not have the time to finish reading something before it is due back in.

\noindent \\ If you could, I would also like there to be pictures for our books! That would make it way more fun,

\paragraph{Interview with Mrs. White, Head Teacher}

\subparagraph{What would you consider your skill level to be regarding technology?}

\noindent \\ I would descibe my skill level as a beginner, I can use Office and check my e-mail but that is it.

\subparagraph{Do you currently have a way to manage inventory?}

\noindent \\ Not particuarly. We have an Excel spreadsheet for general inventory like student iPad allocation and as previously mentioned, the paper-based library book system.

\subparagraph{What aspects of this solution do you like?}

\noindent \\ The library system works okay, but the excel spreadsheet is not very enjoyable to use.

\subparagraph{What aspects of this solution do you dislike?}

\noindent \\ It is not very intuitive to search through as it uses numerical IDs instead of more friendly names. We have to cross-reference the spreadsheet with inventory stickers and IDs from our school management system in order to find out which item or device is being used by who.

\subparagraph{What features would you require in a custom solution?}

\noindent \\ I am aware that the primary focus for this system as noted by staff and students
is to replace our paper-based library records, but I wanted to propose extending the system to
be more generalized. Not to detract from the usefullness of a digital library system,
but I feel like the system would be markedly more useful if we could use it for other school
resources, such as textbooks, keeping track of our student iPads and other more
inventory-focused things like that. I would very much like to replace our Excel spreadsheet with something easier to use!

I would also like to be able to search the system using the names of the items themselves, instead of having to memorize IDs.
Currently, it is not possible in our spreadsheet to look up details of eg. all of our student's iPads.
In order do to this currrently we have to find and type in all the IDs on the stickers on the back of the iPads which is tedious to say the least.

\subparagraph{What features would enhance your experience?}

\noindent \\ It would be nice if we could integrate the existing stickers on our stock, so we don't have to recall and re-number all our textbooks, laptops, iPads in use, as that would be impractical.
The stickers have barcodes on them so I would like to be able to scan the barcode in order to bring up the item instead of having to type it in.

It would also be nice if multiple users could access the system.
Although we are currently only a small school. We are expanding rapidly and will have a significant increase in our headcount next academic year.

Furthermore, as noted by Ella, it would be beneficial to be able to access the system from a mobile device, such as an iPad or laptop, as students and staff are issued with them resepectively.

I would like to be able to deploy this system sometime around the end of this acedemic year so that we can start using and adding items to the system before the end of the academic year.

\end{document}