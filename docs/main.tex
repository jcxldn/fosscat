\documentclass{article}

% Package imports
\usepackage{titlesec}
\usepackage{geometry}
\usepackage{fancyhdr}
\usepackage{graphicx}
\usepackage{hyperref} % \url, https://www.overleaf.com/learn/latex/Hyperlinks

\geometry{a4paper, includeheadfoot, portrait, total={}, top=12.5mm, bottom=12.5mm, left=25.4mm, right=25.4mm}

\graphicspath{ {./images/} }

% Variables
\def\projectname{Computer Science NEA}

\title{\projectname}
\author{James Cahill}
\date{Sepetember 2023}

% Configure fancyHDR page style
% https://tex.stackexchange.com/questions/266911/get-fancyhdr-and-geometry-to-work-nicely
\fancypagestyle{style}{
    \fancyhead{} % clear all header fields
    \fancyhead[HL]{\projectname}
    \fancyhead[HR]{James Cahill}
    \renewcommand{\headrulewidth}{0pt} % Remove header line
}
\pagestyle{style}


\begin{document}


\tableofcontents

\pagebreak

\section{Analysis}

% Background
\subsection{Problem Identification}

\subsubsection{Problem Description}

Popular inventory management solutions are relatively expensive, and may be out
of reach for individuals or small schools.
Inventory systems have numerous benefits for businesses and individuals alike; a business
may choose to track their supply levels where an individual may wish to catelogue their DVD collection. \\

\noindent My goal is to create a web-based application aimed at both businesses and individuals to manage
inventory, with additional modern features such as automatic item re-ordering when stocks are running low.\\

\noindent Traditional inventory management solutions are typically single-user at best, whereas I am to create
a multi-user, collaborative environment.

\subsubsection{Interview}

\subsubsection{Existing similar solutions}

% do 4-5 alternatives

\paragraph{\\InvenTree}
\url{https://inventree.org/}

\subparagraph{\\Overview\\}

InvenTree is an \textbf{open-source} inventory management system, providing \textit{low level stock control and part tracking}.
It uses a Python/Django database backend and provides both a \textbf{web-based interface} as well as a REST API for interacting with other services.
InvenTree also has a powerful plugin system for custom applications and other extensions. \\

\noindent Below is a screenshot of the InvenTree homepage.\\
\includegraphics[width=15cm]{inventree_demo_homepage.png}

\subparagraph{Parts applicable to my solution\\}

- concept is similar (web-based), but I'm doing a different approach.\\
\noindent - not indented for stock control

\pagebreak % temp

\paragraph{\\PartKeepr}
\url{https://partkeepr.org/}\\

\includegraphics[width=15cm]{partkeepr_demo_homepage.png}

\subparagraph{\\Overview\\}

PartKeepr is an open-source inventory management system with a focus on electronic components.
It is designed around four main principles:

\begin{itemize}
    \item Fast Part Searching
    \item Ability to add complete part database
    \item Keeping track of stock
    \item Ease of use
\end{itemize}



\subsubsection{Features to be incorporated into solution}

\subsubsection{Feedback from stakeholders}

\subsection{Requirements}

\subsubsection{Stakeholder requirements}

\subsubsection{Software and hardware requirements}

\subsubsection{Success requirements}

\section{Design}

\subsection{User Interface Design}

\subsubsection{Usability Features}

\subsubsection{Feedback from stakeholder}

\subsection{Modular breakdown}

\subsection{Algorithms}

\subsection{Data Dictionary}

\subsection{Inputs and outputs}

\subsection{Validation}

\subsection{Testing}

\subsubsection{Methods}

\subsubsection{Test Plan}

\section{Implementation}

\subsection{First Iteration}

\section{Testing}

\section{Evaluation}

\end{document}