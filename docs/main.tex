\documentclass{article}

% Package imports
\usepackage{titlesec}
\usepackage{geometry}
\usepackage{fancyhdr}
\usepackage{graphicx}
\usepackage{hyperref} % \url, https://www.overleaf.com/learn/latex/Hyperlinks
\usepackage{outlines} % better itemize
\usepackage{comment}
\usepackage{multirow} % tables
\usepackage{noto} % Google Noto Fonts

\usepackage[british]{datetime2} % load before gitinfo2 to customize
\usepackage[mark=true,grumpy=true]{gitinfo2}

% Redefine \gitMark to customize it
% https://mirror.apps.cam.ac.uk/pub/tex-archive/macros/latex/contrib/gitinfo2/gitinfo2.pdf
\renewcommand{\gitMark}{Branch: \gitBranch\,@\,\gitAbbrevHash{}\,\textbullet{}\,\DTMusedate{gitdate}}

\geometry{a4paper, includeheadfoot, portrait, total={}, top=12.5mm, bottom=12.5mm, left=12.5mm, right=12.5mm}

\graphicspath{ {./images/} }

% Variables
\def\projectname{Inventory Project}

% Override subparagraph with a variant that has no indentation
% https://tex.stackexchange.com/a/392014
\makeatletter
\renewcommand\subparagraph{%
\@startsection{subparagraph}{5}{0pt}%
{3.25ex \@plus 1ex \@minus .2ex}{-1em}%
{\normalfont\normalsize\bfseries}}
\makeatother

\title{\projectname}
\author{James Cahill}
\date{Sepetember 2023}

% Configure fancyHDR page style
% https://tex.stackexchange.com/questions/266911/get-fancyhdr-and-geometry-to-work-nicely
\fancypagestyle{style}{
    \fancyhead{} % clear all header fields
    \fancyhead[HL]{\projectname}
    \fancyhead[HR]{James Cahill}
    \renewcommand{\headrulewidth}{0pt} % Remove header line
}
\pagestyle{style}


\begin{document}


\tableofcontents

\pagebreak

\section{Analysis}

% Background
\subsection{Problem Identification}

\subsubsection{Problem Description}

\begin{comment}
    - not always designed for multi user?
    - not intuitive?

    GS (till end): not user friendly?
    shoudn't be time consuming to add data/etc
    manual admin can be time consuming, try to automate
    ability for mutliple concurrent users
        easy and transpaancy for multi users
                log in and see what was done the day before
                    easy to follow/read

    not able to re-order items in a single place
    have to go do diff platforms when ready to re-order

    limited functionality
        want to be able to track purchased items
        as well as handle budgeting
        existing solutions don't have multiple functions
        so you need to duplicate information on different apps/areas
        like how Monzo combines budgeting with card and tracking txns.

    are these a problem with the current app or a problem you have that
    you need a solution to.

    is problem with current app
    or need support with inventory/organisation

    do more research on downsides of different alternatives/
\end{comment}

Popular inventory management solutions are relatively expensive, and may be out
of reach for individuals or small schools.
Inventory systems have numerous benefits for businesses and individuals alike; a business
may choose to track their supply levels where an individual may wish to catalogue their DVD collection. \\

\noindent My goal is to create a web-based application aimed at both businesses and individuals to manage
inventory, with additional modern features such as automatic item re-ordering when stocks are running low.\\

\noindent Traditional inventory management solutions are typically single-user at best, whereas I intend to create
a multi-user, collaborative environment.\\

\noindent In my view, an inventory system should be:

\begin{outline}
    \1 Easy for end users to use.
    \1 Cross platform
    \1 Performant interface
    \1 Efficient in terms of adding data
    \1 Allow for easy cataloguing of inventory
    \1 Allow for item scanning using QR codes / barcodes
    \1 Be able to source data from external sources
    \1 Support both consumable and non-consumable goods.

    
\end{outline}


\begin{comment}
An inventory system should be able to:

time consuming to add data
not user friendly

- catalogue of inventory, re-order for you
- scan using a phone (no external hardware needed)
- alert / re-order when stocks are running low.
- purchase links
- stretch: source data from amazon or equivalent instead of typing it manually
- search engine for catalogued and new Parts
    - provides with options for where to purchase certain goods
- button to re-order
    - smart device???????
- predict when stocks will run out.
- source data from external sources
- like monzo projection of when it will run out
- how much you are spending each month on goods
- nfc support to easily scan / etc items (might be too hard on iOS)
\textbf{Barcode check in / out}
- monzo integration
- budgeting - figure projections  as well 
clearly define what the APP will feature.
Think about
- potential users
    - how does the app cater to their needs - different features etc


\end{comment}

\subsubsection{Stakeholders}

\newcommand{\stakeholderEntry}[4]{{#1} & {#2} & {#3} & {#4}\\\hline}

\begin{tabular}{ |c|p{0.3\textwidth}|c|c| }
    \hline
    \textbf{Stakeholder} & \textbf{Description} & \textbf{Current Use} & \textbf{Requirements}\\
    \hline

    \stakeholderEntry
        {Claire Foley}
        {SENCo, Library Lead at The Village Prep School}
        {No system. Library books are not tracked.}
        {TBD}
    
    \stakeholderEntry
        {TBD}{Freelance photographer}{Excel Spreadsheet}{TBD}
    \hline
\end{tabular}


\subsubsection{Interview}

\paragraph{Question Set}
\begin{outline}
    \1 What would you consider your skill level to be regarding technology?
    \1 Do you currently have a way to manage inventory?
    \2 If so, what is your current solution?
    \3 What aspects of this solution do you like?
    \3 What aspects of this solution do you dislike?
    \2 What features would you \textbf{require} in a custom solution?
    \2 What features would \textbf{enchance} your experience?
\end{outline}

\begin{comment}
    CF 
CF "SLT": library

    what is the current system?
        borrowing cards and date stamps that are manually written in for when books are due back
        borrower card for who borrowed which book, stickers placed on books for categories

    problems with existing solution
        time consuming; cards get lost; stickers fall off
        not very quick to see who borrowed what; have to look through all cards
        
    
    like about the existing solution
        primary aged students/children can do it (themselves)
    
    requirements for new solution

        scan a barcode / borrower ticket and instantly see what they've borrowed.
        
        cost effective
        compatible with existing hardware/software
        works on ipa
        ability to renew
        notified when books are overdue
        statistics (book numbers) by genre/author/category
    
    enhancements for new solution
        colored stickers for categories
        ability to charge parents (make an invoice?)
    
    is there a specific way you would like the system to be organised?
        yes, we are a library so by preset genres and categories
    
    do you have any questions of your own?
        what's the timeframe for this being completed - March.

fake pupil: Ella (no surname)
year 6 "head librarian" age 10-**11** (y6)

    problems with the current system
        "so old fashioned, should be able to scan using my iPad! Then it could be pupil-led."
    
    benefits
        can see who borrowed the book I want and nag them to return it so I can read it!
        when I forget which book I've borrowed the teacher can easily find which books I've lost
    
    requirements
        iPads; fast (can be done in break times - borrow/renew)
    
    
        

fake teacher
equipment for science lab/art room/ etc


can say helping a junior school.

    
\end{comment}

\pagebreak

\subsubsection{Existing similar solutions}

% do 4-5 alternatives

% 1. InvenTree
\paragraph{\\InvenTree}
\url{https://inventree.org/}

\subparagraph{\\Overview\\}

InvenTree is an \textbf{open-source} inventory management system, providing \textit{low level stock control and part tracking}.
It uses a Python/Django database backend and provides both a \textbf{web-based interface} as well as a REST API for interacting with other services.
InvenTree also has a powerful plugin system for custom applications and other extensions. \\

\noindent Below is a screenshot of the InvenTree homepage.\\
\includegraphics[width=15cm]{inventree_demo_homepage.png}

\subparagraph{Parts applicable to my solution\\}

The core concept is similar (the application is web-based), but my solution will be more generalized that just stock control.

\pagebreak

% 2. PartKeepr
\paragraph{\\PartKeepr}
\url{https://partkeepr.org/}\\

\includegraphics[width=15cm]{partkeepr_demo_homepage.png}

\subparagraph[indent=false]{\\Overview\\}

PartKeepr is an open-source inventory management system with a focus on electronic components.
It is designed around four main principles:

\begin{outline}
    \1 Fast Part Searching
    \1 Ability to add complete part database
    \1 Keeping track of stock
    \1 Ease of use
\end{outline}

\subparagraph{Parts applicable to my solution\\\\}

\noindent Like PartKeepr, I hope to implement a web-based interface.
However, I am using a different approach as my solution will not be tailored specifically to electronic components.

\pagebreak

% 3. Sortly
\paragraph{\\Sortly}
\url{https://www.sortly.com/solutions/inventory-management-software/}\\

\includegraphics[width=15cm]{sortly_homepage_mockup_1.png}

\subparagraph{\\Overview\\}

Sortly is a proprietary cloud-based inventory management system with a focus on small businesses and inviduals.\\\\
It has two plans available, an always free plan with limited functionality and a paid plan will a more complete feature-set.

\subparagraph{Parts applicable to my solution\\\\}

I hope to implement the following features from Sortly:

\begin{outline}
    \1 Web based interface
    \2 Allows for easy access.

    \1 Barcode support
    \2 Allows end users to print off QR codes to stick to items
    \2 Which can be scanned in-app to easily perform actions on the item.

    \1 Real-time reporting insights
    %\subitem \includegraphics[width=10cm]{sortly_homepage_mockup_2.png}
    \2 Allows for added insight into usage patterns for particular units.
\end{outline}

\subsubsection{Features to be incorporated into solution}

\subsubsection{Feedback from stakeholders}

\subsection{Requirements}

\subsubsection{Stakeholder requirements}

\subsubsection{Software and hardware requirements}

\subsubsection{Success requirements}

\section{Design}

\subsection{User Interface Design}

\subsubsection{Usability Features}

\subsubsection{Feedback from stakeholder}

\subsection{Modular breakdown}

\subsection{Algorithms}

\subsection{Data Dictionary}

\subsection{Inputs and outputs}

\subsection{Validation}

\subsection{Testing}

\subsubsection{Methods}

\subsubsection{Test Plan}

\section{Implementation}

\subsection{First Iteration}

\section{Testing}

\section{Evaluation}

\end{document}