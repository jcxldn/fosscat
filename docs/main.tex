\documentclass{article}

% Package imports
\usepackage{titlesec}
\usepackage{geometry}
\usepackage{fancyhdr}
\usepackage{graphicx}
\usepackage{hyperref} % \url, https://www.overleaf.com/learn/latex/Hyperlinks
\usepackage{outlines} % better itemize
\usepackage{comment}
\usepackage{multirow} % tables
\usepackage{noto} % Google Noto Fonts
\usepackage{hyperref} % Hyperlinks to sectiosn

\usepackage[british]{datetime2} % load before gitinfo2 to customize
\usepackage[mark=true,grumpy=true]{gitinfo2}

\usepackage{subfiles} % Best loaded last in the preamble

% Redefine \gitMark to customize it
% https://mirror.apps.cam.ac.uk/pub/tex-archive/macros/latex/contrib/gitinfo2/gitinfo2.pdf
\renewcommand{\gitMark}{Branch: \gitBranch\,@\,\gitAbbrevHash{}\,\textbullet{}\,\DTMusedate{gitdate}}

% Word style normal margins.
\geometry{a4paper, includeheadfoot, portrait, total={}, top=12.5mm, bottom=12.5mm, left=25.4mm, right=25.4mm}

\graphicspath{ {./images/} }

% Variables
\def\projectname{Inventory Project}

% Override subparagraph with a variant that has no indentation
% https://tex.stackexchange.com/a/392014
\makeatletter
\renewcommand\subparagraph{%
\@startsection{subparagraph}{5}{0pt}%
{3.25ex \@plus 1ex \@minus .2ex}{-1em}%
{\normalfont\normalsize\bfseries}}
\makeatother

\title{\projectname}
\author{James Cahill}
\date{Sepetember 2023}

% Configure fancyHDR page style
% https://tex.stackexchange.com/questions/266911/get-fancyhdr-and-geometry-to-work-nicely
\fancypagestyle{style}{
    \fancyhead{} % clear all header fields
    \fancyhead[HL]{\projectname}
    \fancyhead[HR]{James Cahill}
    \renewcommand{\headrulewidth}{0pt} % Remove header line
}
\pagestyle{style}


\begin{document}


\tableofcontents

\pagebreak

\section{Analysis}

% Background
\subsection{Problem Identification}

\subsubsection{Problem Description}
\subfile{sections/analysis/problemIdentification/problemDescription}

\subsubsection{Stakeholders}
\subfile{sections/analysis/problemIdentification/stakeholders}

\subsubsection{Why is it suitable to a computational solution?}

\begin{comment}
why creating this solution is better with the use of technology
eg:
need a way to store large amounts of data; perfect for a database
easy way to add/remove inventory (would be labour intensive otherwise - paper based systems)
can be v. easily done with a gui

identifying key things the solution should have; explain why doing
this computationally is a good idea / is easy

1/2 a page to a page

eg decomposition/abstraction
decomp:
large program; by splitting into smaller sub-programs
can make each one individually and combine at the end
explain how they can be used to achieve the goals/impls
\end{comment}

\subsection{Investigation}

\subsubsection{Preparation for interview}
\subfile{sections/analysis/investigation/prep}

\subsubsection{Interviews}

\begin{comment}
2 or 3

person
question
answer
brief summary
\end{comment}

\subsubsection{Summary of interviews}

\begin{comment}
half a page of key things that were found out from the interviews
should include / should not include / etc.


\end{comment}

\pagebreak

\subsection{Research}


\subsubsection{Existing similar solutions}
\subfile{sections/analysis/research/existing}

\subsubsection{Features to be incorporated into solution}

\begin{comment}
based on research etc
select the features from the research that will be incorporated
and explain what they do
from sortly, steal x feature because y
include things you won't include as well (out of scope), because xyz
\end{comment}

\subsubsection{Limitations of the solution}

\begin{comment}
limitations:

- time

- limited by any software

- money - hosting backend?
- not getting an apple dev account
so won't be a "true" mobile app, more of a website on the home screen.
\end{comment}

\subsubsection{Feedback from stakeholders}

\subsection{Requirements}

\subsubsection{Stakeholder requirements}
\label{sec:stakeholderRequirements}

\pagebreak

\subsubsection{Software and hardware requirements}
\subfile{sections/analysis/requirements/software_hardware}

\pagebreak

\subsubsection{Success requirements}
\subfile{sections/analysis/requirements/success}

\section{Design}

\begin{comment}
design: for each page/screen:

picture of page


brief desc of what the page will do

for each one show the stakeholder requirements or success requirements that will be met when this page/feature is implmented;


then break down each component of the design page.
sentance or two on what it does and why (justify it being there)

\end{comment}

\subsection{User Interface Design}

\subsubsection{Usability Features}

\subsubsection{Feedback from stakeholder}

\subsection{Modular breakdown}

\subsection{Algorithms}

\subsection{Data Dictionary}

\subsection{Inputs and outputs}

\subsection{Validation}

\subsection{Testing}

\subsubsection{Methods}

\subsubsection{Test Plan}

\pagebreak

\section{Implementation}

\subsection{First Iteration | Initial Backend and Database}

\subsubsection{Introduction}

\noindent For my first iteration I decided to work on the backend and the databse.
I am writing the backend in \underline{Go}.
Go is a performant, statically typed high level language designed by Google.
Is is frequently used for backend development thanks to it's performance and memory safety.
I am going to use \underline{GraphQL} as the query language for the frontend to interact with the backend. GraphQL is an open-source query and manipulation language designed for use in APIs. (Application Programming Interface). The backend will serve as an API which will interface with my database.
I choose to use GraphQL as it is better suited for larger, more complex data sources, and supports querying for multiple different types of data at once, unlike REST. It is also something I was interested in learning more about as I have not designed a system using it before.

\noindent\\ The first thing I decided to work on was user account creation. This would involve using GraphQL to receive data about the new user, generating the user ID, and then placing that data in a database.
After doing some research (should I talk about the different options?) I decided to use the following three libraries for my project:

\begin{outline}
    \1 \textbf{Gin}\\
    Gin is a HTTP framework for Go. This allows me to expose my GraphQL endpoint over HTTP. After conducting some research, I choose to use gin as it is by far the fastest library available at the time of writing.
    It offers performance up to 40 times faster than it's closest alternative, \textit{"Martini"}.
    Gin is a very popular framework, with over 72 thousand stars on GitHub. Using this library benefits me as it has been battle-tested by other users and includes end to end testing, making it more than suitable for my project. (wording isn't great, trying to say it has been tested already)

    \1 \textbf{Gqlgen}\\
    Gqlgen is a Go library designed to make building GraphQL servers easy and hassle-free. It is designed with a schema-first approach, meaning the developer (me) can simply define their API using the standard GraphQL Schema Definition Language. It prioritizes type safety, and has rudimentary validation support built-in. However, I will not be relying on this validation feature, I will create my own validation system that better suits my needs.

    \1 \textbf{GORM} (Go Object Relational Mapping library)\\
    GORM is a fully-features ORM library for Go. (explain what an ORM does)
\end{outline}

(skip to 14.11.23, after voice memos)



\section{Testing}

\section{Evaluation}

\end{document}