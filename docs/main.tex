% Incorporates https://gist.github.com/FelipeCortez/10729134

\documentclass[11pt,a4paper]{article}

% Package imports
\usepackage{titlesec}
\usepackage{geometry}
\usepackage{fancyhdr}
\usepackage{graphicx}
\usepackage{hyperref} % \url, https://www.overleaf.com/learn/latex/Hyperlinks
\usepackage{outlines} % better itemize
\usepackage{comment}
\usepackage{multirow} % tables
\usepackage{noto-sans} % Google Noto Fonts
\usepackage{hyperref} % Hyperlinks to sectiosn
\usepackage{listings}
\usepackage{lstautogobble} % listings: Fix relative indenting
\usepackage{color} % listings: Code coloring
\usepackage{zi4} % listings: Nice font
\usepackage{array} % for \begin{tabular}{ | p{'width'} | } (https://www.overleaf.com/learn/latex/Tables)

\usepackage[british]{datetime2} % load before gitinfo2 to customize
\usepackage[mark=true,grumpy=true]{gitinfo2}

\usepackage{tikz}
\usepackage{tikz-uml}

\usepackage{subfiles} % Best loaded last in the preamble

% Redefine \gitMark to customize it
% https://mirror.apps.cam.ac.uk/pub/tex-archive/macros/latex/contrib/gitinfo2/gitinfo2.pdf
\renewcommand{\gitMark}{Branch: \gitBranch\,@\,\gitAbbrevHash{}\,\textbullet{}\,\DTMusedate{gitdate}}

% Use sans font (noto sans) as default font for this document
\renewcommand{\familydefault}{\sfdefault}

% Word style normal margins.
\geometry{a4paper, includeheadfoot, portrait, total={}, top=12.5mm, bottom=12.5mm, left=25.4mm, right=25.4mm}

\graphicspath{ {./images/} }

% Variables
\def\projectname{Inventory Project}

% TiKz style defines for flowcharts
\tikzstyle{startstop} = [ellipse, rounded corners, minimum width=3cm, minimum height=1cm,text centered, draw=black, fill=red!30]
\tikzstyle{io} = [trapezium, trapezium left angle=70, trapezium right angle=110, minimum width=3cm, minimum height=1cm, text centered, draw=black, fill=blue!30]
\tikzstyle{process} = [rectangle, minimum width=3cm, minimum height=1cm, text centered, draw=black, fill=orange!30]
\tikzstyle{decision} = [diamond, minimum width=3cm, minimum height=1cm, text centered, draw=black, fill=green!30]
\tikzstyle{arrow} = [thick,->,>=stealth]

% TiKz style defines for modular breakdown
\tikzstyle{mb_node} = [rectangle, rounded corners, minimum width=3cm, minimum height=1cm,text centered, draw=black]
\tikzstyle{line} = [thick,-,>=stealth]


% Override subparagraph with a variant that has no indentation
% https://tex.stackexchange.com/a/392014
\makeatletter
\renewcommand\subparagraph{%
\@startsection{subparagraph}{5}{0pt}%
{3.25ex \@plus 1ex \@minus .2ex}{-1em}%
{\normalfont\normalsize\bfseries}}
\makeatother

\title{\projectname}
\author{James Cahill}
\date{Sepetember 2023}

% Configure fancyHDR page style
% https://tex.stackexchange.com/questions/266911/get-fancyhdr-and-geometry-to-work-nicely
\fancypagestyle{style}{
    \fancyhead{} % clear all header fields
    \fancyhead[HL]{\projectname}
    \fancyhead[HR]{James Cahill}
    \renewcommand{\headrulewidth}{0pt} % Remove header line
}
\pagestyle{style}

% Configure listings colours
\definecolor{bluekeywords}{rgb}{0.13, 0.13, 1}
\definecolor{greencomments}{rgb}{0, 0.5, 0}
\definecolor{redstrings}{rgb}{0.9, 0, 0}
\definecolor{graynumbers}{rgb}{0.5, 0.5, 0.5}

% Configure listings style
\lstset{
    autogobble,
    columns=fullflexible,
    showspaces=false,
    showtabs=false,
    breaklines=true,
    showstringspaces=false,
    breakatwhitespace=true,
    escapeinside={(*@}{@*)},
    commentstyle=\color{greencomments},
    keywordstyle=\color{bluekeywords},
    stringstyle=\color{redstrings},
    numberstyle=\color{graynumbers},
    basicstyle=\ttfamily\footnotesize,
    frame=l,
    framesep=12pt,
    xleftmargin=12pt,
    tabsize=4,
    captionpos=b
}

\usepackage{xcolor}

% Based on: https://github.com/nextstepitt/standard-course-template/blob/master/Workbook/_Preambles/workbook-def.tex
% SPDX-License-Identifier: MIT
% https://github.com/nextstepitt/standard-course-template

%% JavaScript

\lstdefinelanguage{javascript}{
    keywords=[1]{break, case, catch, class, const, continue, debugger, default, delete, do, else, export, extends, finally, for, function, if, import, in, instanceof, let, new, of, return, static, switch, throw, try, typeof, var, while, var},
    keywordstyle=[1]\color{blue}\bfseries,
    keywords=[2]{Array, boolean, Date, false, implements, import, NaN, Number, Math, null, String, super, this, true, undefined},
    keywordstyle=[2]\color{darkgray}\bfseries,
    keywords=[3]{isNaN, parseInt, parseFloat},
    keywordstyle=[3]\color{darkgray}\bfseries,
    identifierstyle=\color{black},
    sensitive=false,
    comment=[l]{//},
    morecomment=[s]{/*}{*/},
    commentstyle=\color{purple}\ttfamily,
    stringstyle=\color{red}\ttfamily,
    morestring=[b]',
    morestring=[b]"
}

\lstdefinelanguage{js}[]{javascript}{}

%% JSX extends JavaScript

\lstdefinelanguage{jsx}[]{javascript}{
    alsoletter=-,
    morestring=[b]",stringstyle=\color[rgb]{0,0,1},
    moredelim=*[s][{\color[rgb]{0.75,0,0}}]{<}{>},
    moredelim=[s][{\color{purple}}]{<!--}{-->},
    moredelim=[l][{\color[rgb]{0,0,0}}]{\ <\ },
    moredelim=[l][{\color[rgb]{0,0,0}}]{\ <=\ },
    moredelim=[l][{\color[rgb]{0,0,0}}]{\ >\ },
    moredelim=[l][{\color[rgb]{0,0,0}}]{\ >=\ },
    moredelim=[l][{\color[rgb]{0,0,0}}]{\ =>\ },
}

%% JSON looks like JavaScript arrays and objects, but nothing more. There are only strings, numbers, and true and false.

\lstdefinelanguage{json}{
    keywords=[2]{false, true},
    keywordstyle=[2]\color{darkgray}\bfseries,
    stringstyle=\color{red}\ttfamily,
    morestring=[b]',
    morestring=[b]"
}

%% TypeScript extends JavaScript.

\lstdefinelanguage{typescript}[]{javascript}{
    keywords=[91]{interface, enum, implements},
    keywordstyle=[91]\color{blue}\bfseries,
    keywords=[92]{string, number, boolan},
    keywordstyle=[2]\color{darkgray}\bfseries,
}

\lstdefinelanguage{ts}[]{typescript}{}

%% TSX extends TypesScript (this is a copy of jsx, but we cannot combine them).

\lstdefinelanguage{tsx}[]{typescript}{
    alsoletter=-,
    morestring=[b]",stringstyle=\color[rgb]{0,0,1},
    moredelim=*[s][{\color[rgb]{0.75,0,0}}]{<}{>},
    moredelim=[s][{\color{purple}}]{<!--}{-->},
    moredelim=[l][{\color[rgb]{0,0,0}}]{\ <\ },
    moredelim=[l][{\color[rgb]{0,0,0}}]{\ <=\ },
    moredelim=[l][{\color[rgb]{0,0,0}}]{\ >\ },
    moredelim=[l][{\color[rgb]{0,0,0}}]{\ >=\ },
    moredelim=[l][{\color[rgb]{0,0,0}}]{\ =>\ },
}


% Automating file inclusion, initially based on https://en.wikibooks.org/wiki/LaTeX/Source_Code_Listings#Automating_file_inclusion
% Fields have changed from the original
\newcommand{\includecode}[2][c]{
    \lstinputlisting[
        caption=\lstinline{#2},
        captionpos=t, % caption position: top
        escapechar=,
        language=#1,
        frame=single
    ]{#2}
    %\centerline{\rule{13cm}{0.4pt}}
    %\bigskip
}

\lstdefinelanguage{graphql}
{
    morekeywords={type, query, mutation, subscription, interface, union, enum, scalar, schema, directive, fragment, on, input, as, null, true, false},
    sensitive=true,
    morecomment=[l]{\%},
    morecomment=[l]{\#},
    otherkeywords={Int, Float, String, Boolean, ID},
    literate={->}{{$\mapsto$}}2,
    literate={...}{{$\dots$}}2
}



\begin{document}


\tableofcontents

\pagebreak

\section{Analysis}

% Background
\subsection{Problem Identification}

\subsubsection{Problem Description}
\subfile{sections/analysis/problemIdentification/problemDescription}

\subsubsection{Stakeholders}
\subfile{sections/analysis/problemIdentification/stakeholders}

\subsubsection{Why is it suitable to a computational solution?}

\begin{comment}
why creating this solution is better with the use of technology
eg:
need a way to store large amounts of data; perfect for a database
easy way to add/remove inventory (would be labour intensive otherwise - paper based systems)
can be v. easily done with a gui

identifying key things the solution should have; explain why doing
this computationally is a good idea / is easy

1/2 a page to a page

eg decomposition/abstraction
decomp:
large program; by splitting into smaller sub-programs
can make each one individually and combine at the end
explain how they can be used to achieve the goals/impls
\end{comment}

\subsection{Investigation}

\subsubsection{Preparation for interview}
\subfile{sections/analysis/investigation/prep}

\subsubsection{Interviews}

\begin{comment}
2 or 3

person
question
answer
brief summary
\end{comment}

\subsubsection{Summary of interviews}

\begin{comment}
half a page of key things that were found out from the interviews
should include / should not include / etc.


\end{comment}

\pagebreak

\subsection{Research}


\subsubsection{Existing similar solutions}
\subfile{sections/analysis/research/existing}

\subsubsection{Features to be incorporated into solution}

\begin{comment}
based on research etc
select the features from the research that will be incorporated
and explain what they do
from sortly, steal x feature because y
include things you won't include as well (out of scope), because xyz
\end{comment}

\subsubsection{Limitations of the solution}

\begin{comment}
limitations:

- time

- limited by any software

- money - hosting backend?
- not getting an apple dev account
so won't be a "true" mobile app, more of a website on the home screen.
\end{comment}

\subsubsection{Feedback from stakeholders}

\subsection{Requirements}

\subsubsection{Stakeholder requirements}
\label{sec:stakeholderRequirements}

\pagebreak

\subsubsection{Software and hardware requirements}
\subfile{sections/analysis/requirements/software_hardware}

\pagebreak

\subsubsection{Success requirements}
\subfile{sections/analysis/requirements/success}

\section{Design}

\begin{comment}
design: for each page/screen:

picture of page


brief desc of what the page will do

for each one show the stakeholder requirements or success requirements that will be met when this page/feature is implmented;


then break down each component of the design page.
sentance or two on what it does and why (justify it being there)

\end{comment}

\subsection{User Interface Design}

\subsubsection{Usability Features}

\subsubsection{Feedback from stakeholder}

\subsection{Modular breakdown}
\subfile{sections/design/modular_breakdown.tex}

\subsection{Algorithms}
\subfile{sections/design/algorithms.tex}

\subsection{Data Dictionary}

\subsubsection{Classes}

\subsection{Inputs and outputs}

\subsection{Validation}

\subsection{Testing}

\subsubsection{Methods}

\subsubsection{Test Plan}

\pagebreak

\section{Implementation}

\subsection{First Iteration | Initial Backend and Database}
\subfile{sections/implementation/first_iteration.tex}


\pagebreak

\section{Testing}

\section{Evaluation}

\section{Code Listings}
\subfile{sections/code.tex} % \document 

\end{document}